\documentclass[a4paper,12pt]{article}
\usepackage{float}
\usepackage[table]{xcolor} % Para colorir as linhas da tabela
%\usepackage{array} % Para melhorar a formatação de tabelas
\usepackage[brazil]{babel}
\usepackage[utf8]{inputenc}
\usepackage{diagbox}
\usepackage{amsmath}
\usepackage{amssymb}
\usepackage{multicol}
\usepackage{graphicx}
%\usepackage{animate}
%\usepackage{tabularx}
\usepackage[font=footnotesize, labelfont=bf, listformat=empty]{caption}
\usepackage{geometry}
\usepackage{listings}
\usepackage{fontspec}
\usepackage{minted}
\usepackage{lastpage}
%\usepackage{subcaption}
\usepackage[skins,xparse,breakable]{tcolorbox}
%\usepackage{longtable}
\usepackage{titlesec}
\usepackage{fancyhdr}
\usepackage{setspace}
\usepackage{booktabs}
\usepackage{enumitem}
\usepackage{minted}
\usepackage[colorlinks=true, allcolors=black]{hyperref}


%\usetikzlibrary{arrows,shapes,automata,positioning,calc}
% Configurações de margens e espaçamento
\geometry{a4paper, left=2.5cm, right=2.5cm, top=2.5cm, bottom=2cm}
\onehalfspacing

% Configurações de títulos
\titleformat{\section}{\normalfont\Large\bfseries}{\thesection}{1em}{}
\titleformat{\subsection}{\normalfont\large\bfseries}{\thesubsection}{1em}{}
\titleformat{\subsubsection}
  {\normalfont\large}
  {\textbf{\thesubsubsection}}
  {0.25em}{}

\setcounter{tocdepth}{3}
\newcommand{\cod}[1]{\texttt{#1}}

\renewcommand{\listingscaption}{Código}
\newenvironment{code}{\captionsetup{type=listing}}{}

% Configurações de cores
\definecolor{LightSeaGreen}{rgb}{0.1255, 0.6980, 0.6667}
\definecolor{whitesmoke}{rgb}{0.96, 0.96, 0.96}
\definecolor{cinza}{RGB}{160, 160, 160}
\definecolor{cinza_claro}{RGB}{240, 240, 240} % Cinza claro

\newtcolorbox{codebox}[1]{
    title=#1,
    fonttitle=\small\bfseries,
    colback=blue!2.5!white,
    colframe=blue!62!white,
    arc=2pt,
    boxrule=0.8pt,
    left=4pt,
    right=4pt,
    top=4pt,
    bottom=4pt,
    breakable,
    enhanced jigsaw
}

\setminted[asm]{
    fontsize=\scriptsize,
    style=manni,
    linenos=true,
    breaklines=true,
    mathescape=true,
    bgcolor=cinza_claro,
    breakanywhere=true,
}

\newcommand{\hex}[1]{\textcolor{green!50!black}{#1}}

% Configurações de cabeçalho e rodapé
\pagestyle{fancy}
\fancyhf{}
\fancyhead[L]{\footnotesize{Relatório do Laboratório 2}}
\fancyhead[R]{\footnotesize{2025/2 - Turma 02}}
\fancyfoot[R]{\footnotesize{Página \hspace{0.05cm} \thepage \hspace{0.05cm} de \pageref{LastPage}}}
\fancyfoot[L]{\footnotesize{Organização e Arquitetura de Computadores - CIC0099}}

% Adiciona uma linha preta acima do rodapé
\renewcommand{\footrulewidth}{0.4pt} % Espessura da linha
\renewcommand{\footrule}{\vbox to 0pt{\hrule width \textwidth height \footrulewidth \vss}} % Desenha a linha

% Novo Comando para chamar a capa
\newcommand{\capa}{
    \begin{titlepage}
        \begin{multicols}{2}
            \begin{flushleft}
                \includegraphics[width=0.45\linewidth]{Recursos/Imagens/UnB_logo.png}
            \end{flushleft}
            \columnbreak
            \begin{flushright}
                Universidade de Brasília \\
                Instituto de Exatas \\
                Departamento de Ciência da Computação
            \end{flushright}
        \end{multicols}
        \begin{center}
        \vspace{-20pt}
        \rule{\textwidth}{0.4pt}
        \end{center}
        \vspace{0.6cm}
        \begin{center}
            {\Huge \textbf{Relatório do Laboratório 2}} \\[0.5em]
            {\Huge \textbf{Processador Uniciclo}} \\ [0.5em]
            {\large CIC0099 - Organização e Arquitetura de Computadores} \\[0.25em]
            {\large Turma 02 - Prof. Dr. Ricardo Pezzuol Jacobi} \\
            \vfill
            {\Large \textbf{Grupo B4}} \\
            \begin{table}[H]
                \centering
                \begin{tabular}{l r}
                    Henrique Morcelles Salum & 232003008 \\
                    Athos Calixto Muniz & 211068261 \\
                    Gabriela Fernanda Rodrigues Costa & 180120859 \\
                    Cauê Araújo Euzebio & 211028195 \\
                    Lucas Silva Nóbrega & 180035096
                \end{tabular}
                \label{tab:placeholder}
            \end{table}
        \end{center}
    \end{titlepage}
}

\begin{document}

\capa

% Sumário
\newpage
\tableofcontents
\newpage

\section{Introdução}
O presente experimento tem como objetivo introduzir os alunos ao processo de projeto e implementação de sistemas digitais utilizando VHDL como Linguagem de Descrição de Hardware (HDL). A atividade busca promover o treinamento prático em modelagem, análise e síntese de circuitos digitais, utilizando o ambiente de desenvolvimento \textbf{Intel Quartus Prime v24.1} como ferramenta principal.

Como aplicação prática, propõe-se a implementação de um processador Uniciclo compatível com a ISA RV32I reduzida, contendo as instruções:\cod{add}, \cod{sub}, \cod{and}, \cod{or}, \cod{slt}, \cod{lw}, \cod{sw}, \cod{beq}, \cod{jal}, \cod{jalr}, \cod{addi} e \cod{lui}. Serão desenvolvidos os blocos fundamentais do processador, como banco de registradores com três portas de leitura, gerador de imediatos, ULA e unidade de controle.

A implementação será validada por meio de simulação funcional e temporal, com análise de \textit{timing} e frequência máxima de operação, permitindo compreender de forma integrada o funcionamento de uma arquitetura RISC-V.


\section{Métodos e Análise}

\subsection{Experimento}
\begin{tcolorbox}[title=Enunciado, colback=blue!5!white, colframe=blue!75!black]
Implemente o processador Uniciclo com ISA Reduzida com as instruções: \cod{add}, \cod{sub}, \cod{and}, \cod{or}, \cod{slt}, \cod{lw},
\cod{sw}, \cod{beq}, \cod{jal}, e ainda as instruções \cod{jalr}, \cod{addi} e \cod{lui}.
\end{tcolorbox}
Para realizar o solicitado, é-nos fornecido o esquemático de uma implementação do Uniciclo com ISA reduzida. Nesse esquemático, porém, foi necessário realizar alterações no caminho de dados, que serão devidamente justificadas adiante.

\begin{figure}[H]
    \centering
    % Important: If latex complains about unicode characters,
% please use "\usepackage[utf8x]{inputenc}" in your preamble
% You can change the size of the picture by putting it into the construct:
% 1) \resizebox{10cm}{!}{"below picture"} to scale horizontally to 10 cm
% 2) \resizebox{!}{15cm}{"below picture"} to scale vertically to 15 cm
% 3) \resizebox{10cm}{15cm}{"below picture"} a combination of above two
% It is not recomended to use the scale option of the tikzpicture environment.
\scalebox{0.35}{
\begin{tikzpicture}[x=1pt,y=-1pt,line cap=rect]
\def\logisimfontA#1{{\fontspec{CMU Serif}#1}} % cmr equivalent
\def\logisimfontB#1{{\fontspec{CMU Bright}#1}}
\def\logisimfontC#1{{\fontspec{CMU Bright}#1}}
\def\logisimfontD#1{{\fontspec{CMU Typewriter Text}#1}} % cmtt equivalent
\definecolor{custcol_87_87_87}{RGB}{135, 135, 135}
\definecolor{custcol_0_0_0}{RGB}{0, 0, 0}
\definecolor{custcol_ff_ff_ff}{RGB}{255, 255, 255}
\draw [line width=4.0pt, custcol_0_0_0 ]  (925.0,335.0) -- (975.0,335.0) ;
\draw [line width=3.0pt, custcol_0_0_0 ]  (895.0,445.0) -- (895.0,455.0) ;
\draw [line width=4.0pt, custcol_0_0_0 ]  (105.0,35.0) -- (225.0,35.0) ;
\draw [line width=3.0pt, custcol_0_0_0 ]  (475.0,185.0) -- (475.0,195.0) ;
\draw [line width=3.0pt, custcol_0_0_0 ]  (995.0,345.0) -- (995.0,375.0) ;
\draw [line width=4.0pt, custcol_0_0_0 ]  (745.0,285.0) -- (745.0,375.0) ;
\draw [line width=4.0pt, custcol_0_0_0 ]  (325.0,285.0) -- (435.0,285.0) ;
\draw [line width=4.0pt, custcol_0_0_0 ]  (325.0,205.0) -- (435.0,205.0) ;
\draw [line width=4.0pt, custcol_0_0_0 ]  (325.0,245.0) -- (435.0,245.0) ;
\draw [line width=3.0pt, custcol_0_0_0 ]  (25.0,495.0) -- (55.0,495.0) ;
\draw [line width=4.0pt, custcol_0_0_0 ]  (265.0,245.0) -- (285.0,245.0) -- (305.0,245.0) ;
\draw [line width=4.0pt, custcol_0_0_0 ]  (225.0,75.0) -- (105.0,75.0) -- (105.0,115.0) -- (105.0,245.0) -- (145.0,245.0) ;
\draw [line width=4.0pt, custcol_0_0_0 ]  (85.0,245.0) -- (105.0,245.0) ;
\draw [line width=4.0pt, custcol_0_0_0 ]  (755.0,265.0) -- (785.0,265.0) ;
\draw [line width=4.0pt, custcol_0_0_0 ]  (515.0,305.0) -- (545.0,305.0) -- (545.0,405.0) -- (805.0,405.0) ;
\draw [line width=4.0pt, custcol_0_0_0 ]  (595.0,155.0) -- (565.0,155.0) -- (565.0,345.0) -- (565.0,425.0) -- (525.0,425.0) ;
\draw [line width=3.0pt, custcol_0_0_0 ]  (495.0,335.0) -- (495.0,345.0) ;
\draw [line width=3.0pt, custcol_0_0_0 ]  (895.0,215.0) -- (895.0,225.0) ;
\draw [line width=4.0pt, custcol_0_0_0 ]  (545.0,305.0) -- (595.0,305.0) ;
\draw [line width=3.0pt, custcol_0_0_0 ]  (835.0,215.0) -- (835.0,225.0) ;
\draw [line width=4.0pt, custcol_0_0_0 ]  (635.0,325.0) -- (675.0,325.0) ;
\draw [line width=4.0pt, custcol_0_0_0 ]  (105.0,115.0) -- (595.0,115.0) ;
\draw [line width=3.0pt, custcol_0_0_0 ]  (65.0,285.0) -- (65.0,305.0) ;
\draw [line width=3.0pt, custcol_0_0_0 ]  (885.0,155.0) -- (995.0,155.0) -- (995.0,105.0) ;
\draw [line width=3.0pt, custcol_0_0_0 ]  (615.0,355.0) -- (615.0,375.0) ;
\draw [line width=3.0pt, custcol_0_0_0 ]  (745.0,245.0) -- (745.0,175.0) -- (765.0,175.0) ;
\draw [line width=3.0pt, custcol_0_0_0 ]  (745.0,155.0) -- (765.0,155.0) ;
\draw [line width=4.0pt, custcol_0_0_0 ]  (975.0,95.0) -- (695.0,95.0) -- (695.0,135.0) -- (635.0,135.0) ;
\draw [line width=4.0pt, custcol_0_0_0 ]  (325.0,325.0) -- (355.0,325.0) ;
\draw [line width=4.0pt, custcol_0_0_0 ]  (325.0,165.0) -- (355.0,165.0) ;
\draw [line width=4.0pt, custcol_0_0_0 ]  (565.0,345.0) -- (595.0,345.0) ;
\draw [line width=4.0pt, custcol_0_0_0 ]  (515.0,225.0) -- (605.0,225.0) -- (605.0,205.0) -- (675.0,205.0) ;
\draw [line width=4.0pt, custcol_0_0_0 ]  (285.0,245.0) -- (285.0,425.0) -- (425.0,425.0) ;
\draw [line width=4.0pt, custcol_0_0_0 ]  (975.0,295.0) -- (955.0,295.0) -- (955.0,485.0) -- (785.0,485.0) -- (785.0,265.0) -- (805.0,265.0) ;
\draw [line width=4.0pt, custcol_0_0_0 ]  (435.0,325.0) -- (405.0,325.0) -- (405.0,505.0) -- (1035.0,505.0) -- (1035.0,315.0) -- (1015.0,315.0) ;
\draw [line width=4.0pt, custcol_0_0_0 ]  (35.0,245.0) -- (5.0,245.0) -- (5.0,5.0) -- (1035.0,5.0) -- (1035.0,75.0) -- (1015.0,75.0) ;
\draw [line width=4.0pt, custcol_0_0_0 ]  (265.0,55.0) -- (975.0,55.0) ;
\fill [line width=4.0pt, custcol_0_0_0]  (285.0,245.0) ellipse (6.0 and 6.0 );
\fill [line width=4.0pt, custcol_0_0_0]  (105.0,245.0) ellipse (6.0 and 6.0 );
\fill [line width=4.0pt, custcol_0_0_0]  (565.0,345.0) ellipse (6.0 and 6.0 );
\fill [line width=4.0pt, custcol_0_0_0]  (785.0,265.0) ellipse (6.0 and 6.0 );
\fill [line width=4.0pt, custcol_0_0_0]  (105.0,115.0) ellipse (6.0 and 6.0 );
\fill [line width=4.0pt, custcol_0_0_0]  (545.0,305.0) ellipse (6.0 and 6.0 );
\draw [line width=4.0pt, custcol_0_0_0 ]  (145.0,135.0) -- (264.0,135.0) ;
\draw [line width=4.0pt, custcol_0_0_0 ]  (265.0,135.0) -- (265.0,354.0) ;
\draw [line width=4.0pt, custcol_0_0_0 ]  (265.0,355.0) -- (146.0,355.0) ;
\draw [line width=4.0pt, custcol_0_0_0 ]  (145.0,355.0) -- (145.0,136.0) ;
\logisimfontB{\fontsize{20pt}{20pt}\bfseries\selectfont\node[inner sep=0, outer sep=0, custcol_0_0_0, anchor=base west] at  (152.0,245.0)  {Instruction};}
\logisimfontB{\fontsize{20pt}{20pt}\bfseries\selectfont\node[inner sep=0, outer sep=0, custcol_0_0_0, anchor=base west] at  (165.0,263.0)  {Memory};}
\fill [line width=1.0pt, custcol_0_0_0]  (145.0,245.0) ellipse (2.0 and 2.0 );
\fill [line width=1.0pt, custcol_0_0_0]  (265.0,245.0) ellipse (2.0 and 2.0 );
\draw [line width=3.0pt, custcol_0_0_0 ]  (325.0,165.0) -- (311.0,165.0) ;
\draw [line width=3.0pt, custcol_0_0_0 ]  (325.0,205.0) -- (311.0,205.0) ;
\draw [line width=3.0pt, custcol_0_0_0 ]  (325.0,245.0) -- (311.0,245.0) ;
\draw [line width=3.0pt, custcol_0_0_0 ]  (325.0,285.0) -- (311.0,285.0) ;
\draw [line width=3.0pt, custcol_0_0_0 ]  (325.0,325.0) -- (311.0,325.0) ;
\draw [line width=5.0pt, custcol_0_0_0 ]  (306.0,245.0) -- (311.0,245.0) ;
\draw [line width=5.0pt, custcol_0_0_0 ]  (311.0,166.0) -- (311.0,324.0) ;
\logisimfontA{\fontsize{10pt}{10pt}\selectfont\node[inner sep=0, outer sep=0, custcol_0_0_0, anchor=base west] at  (314.0,162.0)  {6-0};}
\logisimfontA{\fontsize{10pt}{10pt}\selectfont\node[inner sep=0, outer sep=0, custcol_0_0_0, anchor=base west] at  (314.0,202.0)  {19-15};}
\logisimfontA{\fontsize{10pt}{10pt}\selectfont\node[inner sep=0, outer sep=0, custcol_0_0_0, anchor=base west] at  (314.0,242.0)  {24-20};}
\logisimfontA{\fontsize{10pt}{10pt}\selectfont\node[inner sep=0, outer sep=0, custcol_0_0_0, anchor=base west] at  (314.0,282.0)  {11-7};}
\logisimfontA{\fontsize{10pt}{10pt}\selectfont\node[inner sep=0, outer sep=0, custcol_0_0_0, anchor=base west] at  (314.0,322.0)  {30,14-12};}
\fill [line width=5.0pt, custcol_0_0_0]  (305.0,245.0) ellipse (2.0 and 2.0 );
\fill [line width=5.0pt, custcol_0_0_0]  (325.0,165.0) ellipse (2.0 and 2.0 );
\fill [line width=5.0pt, custcol_0_0_0]  (325.0,205.0) ellipse (2.0 and 2.0 );
\fill [line width=5.0pt, custcol_0_0_0]  (325.0,245.0) ellipse (2.0 and 2.0 );
\fill [line width=5.0pt, custcol_0_0_0]  (325.0,285.0) ellipse (2.0 and 2.0 );
\fill [line width=5.0pt, custcol_0_0_0]  (325.0,325.0) ellipse (2.0 and 2.0 );
\draw [line width=4.0pt, custcol_0_0_0 ]  (595.0,285.0) -- (595.0,365.0) -- (635.0,345.0) -- (635.0,305.0) -- cycle;
\logisimfontC{\fontsize{14pt}{14pt}\selectfont\node[inner sep=0, outer sep=0, custcol_87_87_87, anchor=base west] at  (602.0,310.0)  {0};}
\logisimfontB{\fontsize{12pt}{12pt}\bfseries\selectfont\node[inner sep=0, outer sep=0, custcol_0_0_0, anchor=base west] at  (601.0,328.0)  {Mux};}
\fill [line width=1.0pt, custcol_0_0_0]  (595.0,305.0) ellipse (2.0 and 2.0 );
\fill [line width=1.0pt, custcol_0_0_0]  (595.0,345.0) ellipse (2.0 and 2.0 );
\fill [line width=1.0pt, custcol_0_0_0]  (615.0,355.0) ellipse (2.0 and 2.0 );
\fill [line width=1.0pt, custcol_0_0_0]  (635.0,325.0) ellipse (2.0 and 2.0 );
\logisimfontB{\fontsize{10pt}{10pt}\bfseries\selectfont\node[inner sep=0, outer sep=0, custcol_0_0_0, anchor=base west] at  (597.0,390.0)  {ALUSrc};}
\draw [line width=2.0pt, custcol_0_0_0 ]  (594.0,379.0) -- (605.0,379.0) -- (615.0,375.0) -- (625.0,379.0) -- (637.0,379.0) -- (637.0,395.0) -- (594.0,395.0) -- cycle;
\fill [line width=2.0pt, custcol_0_0_0]  (615.0,375.0) ellipse (2.0 and 2.0 );
\draw [line width=4.0pt, custcol_0_0_0 ]  (975.0,275.0) -- (975.0,355.0) -- (1015.0,335.0) -- (1015.0,295.0) -- cycle;
\logisimfontC{\fontsize{14pt}{14pt}\selectfont\node[inner sep=0, outer sep=0, custcol_87_87_87, anchor=base west] at  (982.0,300.0)  {0};}
\logisimfontB{\fontsize{12pt}{12pt}\bfseries\selectfont\node[inner sep=0, outer sep=0, custcol_0_0_0, anchor=base west] at  (981.0,318.0)  {Mux};}
\fill [line width=1.0pt, custcol_0_0_0]  (975.0,295.0) ellipse (2.0 and 2.0 );
\fill [line width=1.0pt, custcol_0_0_0]  (975.0,335.0) ellipse (2.0 and 2.0 );
\fill [line width=1.0pt, custcol_0_0_0]  (995.0,345.0) ellipse (2.0 and 2.0 );
\fill [line width=1.0pt, custcol_0_0_0]  (1015.0,315.0) ellipse (2.0 and 2.0 );
\logisimfontB{\fontsize{10pt}{10pt}\bfseries\selectfont\node[inner sep=0, outer sep=0, custcol_0_0_0, anchor=base west] at  (971.0,390.0)  {Mem2Reg};}
\draw [line width=2.0pt, custcol_0_0_0 ]  (968.0,379.0) -- (985.0,379.0) -- (995.0,375.0) -- (1005.0,379.0) -- (1022.0,379.0) -- (1022.0,395.0) -- (968.0,395.0) -- cycle;
\fill [line width=2.0pt, custcol_0_0_0]  (995.0,375.0) ellipse (2.0 and 2.0 );
\draw [line width=3.0pt, custcol_0_0_0 ]  (225.0,15.0) -- (225.0,50.0) -- (235.0,55.0) -- (225.0,60.0) -- (225.0,95.0) -- (265.0,75.0) -- (265.0,35.0) -- cycle;
\logisimfontB{\fontsize{20pt}{20pt}\bfseries\selectfont\node[inner sep=0, outer sep=0, custcol_0_0_0, anchor=base west] at  (247.0,60.0)  {+};}
\fill [line width=1.0pt, custcol_0_0_0]  (225.0,35.0) ellipse (2.0 and 2.0 );
\fill [line width=1.0pt, custcol_0_0_0]  (225.0,75.0) ellipse (2.0 and 2.0 );
\fill [line width=1.0pt, custcol_0_0_0]  (265.0,55.0) ellipse (2.0 and 2.0 );
\logisimfontD{\fontsize{14pt}{14pt}\selectfont\node[inner sep=0, outer sep=0, custcol_0_0_0, anchor=base west] at  (24.0,40.0)  {\textcolor{blue}{\texttt{0x00000004}}};}
\fill [line width=1.0pt, custcol_0_0_0]  (105.0,35.0) ellipse (2.0 and 2.0 );
\logisimfontB{\fontsize{10pt}{10pt}\bfseries\selectfont\node[inner sep=0, outer sep=0, custcol_0_0_0, anchor=base west] at  (809.0,208.0)  {WriteMem};}
\draw [line width=2.0pt, custcol_0_0_0 ]  (806.0,195.0) -- (864.0,195.0) -- (864.0,212.0) -- (845.0,212.0) -- (835.0,216.0) -- (825.0,212.0) -- (806.0,212.0) -- cycle;
\fill [line width=2.0pt, custcol_0_0_0]  (835.0,215.0) ellipse (2.0 and 2.0 );
\draw [line width=4.0pt, custcol_0_0_0 ]  (435.0,195.0) -- (514.0,195.0) ;
\draw [line width=4.0pt, custcol_0_0_0 ]  (515.0,195.0) -- (515.0,334.0) ;
\draw [line width=4.0pt, custcol_0_0_0 ]  (515.0,335.0) -- (436.0,335.0) ;
\draw [line width=4.0pt, custcol_0_0_0 ]  (435.0,335.0) -- (435.0,196.0) ;
\logisimfontB{\fontsize{16pt}{16pt}\bfseries\selectfont\node[inner sep=0, outer sep=0, custcol_0_0_0, anchor=base west] at  (443.0,265.0)  {Register};}
\logisimfontB{\fontsize{16pt}{16pt}\bfseries\selectfont\node[inner sep=0, outer sep=0, custcol_0_0_0, anchor=base west] at  (456.0,280.0)  {Bank};}
\draw [line width=1.0pt, custcol_0_0_0 ]  (490.0,334.0) -- (495.0,315.0) -- (501.0,333.0) ;
\fill [line width=1.0pt, custcol_0_0_0]  (435.0,205.0) ellipse (2.0 and 2.0 );
\fill [line width=1.0pt, custcol_0_0_0]  (435.0,245.0) ellipse (2.0 and 2.0 );
\fill [line width=1.0pt, custcol_0_0_0]  (435.0,285.0) ellipse (2.0 and 2.0 );
\fill [line width=1.0pt, custcol_0_0_0]  (435.0,325.0) ellipse (2.0 and 2.0 );
\fill [line width=1.0pt, custcol_0_0_0]  (475.0,195.0) ellipse (2.0 and 2.0 );
\fill [line width=1.0pt, custcol_0_0_0]  (495.0,335.0) ellipse (2.0 and 2.0 );
\fill [line width=1.0pt, custcol_0_0_0]  (515.0,225.0) ellipse (2.0 and 2.0 );
\fill [line width=1.0pt, custcol_0_0_0]  (515.0,305.0) ellipse (2.0 and 2.0 );
\logisimfontB{\fontsize{10pt}{10pt}\bfseries\selectfont\node[inner sep=0, outer sep=0, custcol_0_0_0, anchor=base west] at  (360.0,329.0)  {FUNCT};}
\draw [line width=2.0pt, custcol_0_0_0 ]  (359.0,316.0) -- (400.0,316.0) -- (400.0,334.0) -- (359.0,334.0) -- (355.0,325.0) -- cycle;
\fill [line width=2.0pt, custcol_0_0_0]  (355.0,325.0) ellipse (2.0 and 2.0 );
\logisimfontB{\fontsize{10pt}{10pt}\bfseries\selectfont\node[inner sep=0, outer sep=0, custcol_0_0_0, anchor=base west] at  (360.0,169.0)  {OPCODE};}
\draw [line width=2.0pt, custcol_0_0_0 ]  (359.0,156.0) -- (408.0,156.0) -- (408.0,174.0) -- (359.0,174.0) -- (355.0,165.0) -- cycle;
\fill [line width=2.0pt, custcol_0_0_0]  (355.0,165.0) ellipse (2.0 and 2.0 );
\logisimfontB{\fontsize{10pt}{10pt}\bfseries\selectfont\node[inner sep=0, outer sep=0, custcol_0_0_0, anchor=base west] at  (451.0,178.0)  {WriteReg};}
\draw [line width=2.0pt, custcol_0_0_0 ]  (448.0,165.0) -- (502.0,165.0) -- (502.0,182.0) -- (485.0,182.0) -- (475.0,186.0) -- (465.0,182.0) -- (448.0,182.0) -- cycle;
\fill [line width=2.0pt, custcol_0_0_0]  (475.0,185.0) ellipse (2.0 and 2.0 );
\logisimfontB{\fontsize{12pt}{12pt}\bfseries\selectfont\node[inner sep=0, outer sep=0, custcol_0_0_0, anchor=base west] at  (482.0,363.0)  {CLK};}
\draw [line width=2.0pt, custcol_0_0_0 ]  (479.0,349.0) -- (485.0,349.0) -- (495.0,345.0) -- (505.0,349.0) -- (512.0,349.0) -- (512.0,368.0) -- (479.0,368.0) -- cycle;
\fill [line width=2.0pt, custcol_0_0_0]  (495.0,345.0) ellipse (2.0 and 2.0 );
\logisimfontB{\fontsize{12pt}{12pt}\bfseries\selectfont\node[inner sep=0, outer sep=0, custcol_0_0_0, anchor=base west] at  (884.0,473.0)  {CLK};}
\draw [line width=2.0pt, custcol_0_0_0 ]  (879.0,459.0) -- (885.0,459.0) -- (895.0,455.0) -- (905.0,459.0) -- (912.0,459.0) -- (912.0,478.0) -- (879.0,478.0) -- cycle;
\fill [line width=2.0pt, custcol_0_0_0]  (895.0,455.0) ellipse (2.0 and 2.0 );
\draw [line width=1.0pt, custcol_0_0_0 ]  (35.0,205.0) -- (84.0,205.0) ;
\draw [line width=1.0pt, custcol_0_0_0 ]  (85.0,205.0) -- (85.0,284.0) ;
\draw [line width=1.0pt, custcol_0_0_0 ]  (85.0,285.0) -- (36.0,285.0) ;
\draw [line width=1.0pt, custcol_0_0_0 ]  (35.0,285.0) -- (35.0,206.0) ;
\draw [line width=1.0pt, custcol_0_0_0 ]  (59.0,285.0) -- (65.0,274.0) -- (72.0,285.0) ;
\logisimfontB{\fontsize{20pt}{20pt}\bfseries\selectfont\node[inner sep=0, outer sep=0, custcol_0_0_0, anchor=base west] at  (47.0,252.0)  {PC};}
\draw [line width=1.0pt, custcol_0_0_0 ]  (85.0,210.0) -- (90.0,210.0) -- (90.0,290.0) -- (39.0,290.0) -- (39.0,285.0) ;
\draw [line width=1.0pt, custcol_0_0_0 ]  (91.0,213.0) -- (94.0,213.0) -- (94.0,294.0) -- (43.0,294.0) -- (43.0,291.0) ;
\fill [line width=1.0pt, custcol_0_0_0]  (35.0,245.0) ellipse (2.0 and 2.0 );
\fill [line width=1.0pt, custcol_0_0_0]  (65.0,285.0) ellipse (2.0 and 2.0 );
\fill [line width=1.0pt, custcol_0_0_0]  (85.0,245.0) ellipse (2.0 and 2.0 );
\logisimfontB{\fontsize{12pt}{12pt}\bfseries\selectfont\node[inner sep=0, outer sep=0, custcol_0_0_0, anchor=base west] at  (52.0,321.0)  {CLK};}
\draw [line width=2.0pt, custcol_0_0_0 ]  (49.0,309.0) -- (55.0,309.0) -- (65.0,305.0) -- (75.0,309.0) -- (82.0,309.0) -- (82.0,328.0) -- (49.0,328.0) -- cycle;
\fill [line width=2.0pt, custcol_0_0_0]  (65.0,305.0) ellipse (2.0 and 2.0 );
\logisimfontB{\fontsize{12pt}{12pt}\bfseries\selectfont\node[inner sep=0, outer sep=0, custcol_0_0_0, anchor=base west] at  (60.0,499.0)  {CLK};}
\draw [line width=2.0pt, custcol_0_0_0 ]  (59.0,485.0) -- (88.0,485.0) -- (88.0,506.0) -- (59.0,506.0) -- (59.0,505.0) -- (55.0,495.0) -- cycle;
\fill [line width=2.0pt, custcol_0_0_0]  (55.0,495.0) ellipse (2.0 and 2.0 );
\draw [line width=2.0pt, custcol_0_0_0 ]  (5.0,485.0) -- (24.0,485.0) ;
\draw [line width=2.0pt, custcol_0_0_0 ]  (25.0,485.0) -- (25.0,504.0) ;
\draw [line width=2.0pt, custcol_0_0_0 ]  (25.0,505.0) -- (6.0,505.0) ;
\draw [line width=2.0pt, custcol_0_0_0 ]  (5.0,505.0) -- (5.0,486.0) ;
\draw [line width=2.0pt, custcol_0_0_0 ]  (9.0,495.0) -- (9.0,491.0) -- (15.0,491.0) -- (15.0,499.0) -- (21.0,499.0) -- (21.0,495.0) ;
\fill [line width=2.0pt, custcol_0_0_0]  (25.0,495.0) ellipse (2.0 and 2.0 );
\draw [line width=4.0pt, custcol_0_0_0]  (475.0,425.0) ellipse (49.5 and 50.0 );
\logisimfontB{\fontsize{16pt}{16pt}\bfseries\selectfont\node[inner sep=0, outer sep=0, custcol_0_0_0, anchor=base west] at  (435.0,425.0)  {Immediate};}
\logisimfontB{\fontsize{16pt}{16pt}\bfseries\selectfont\node[inner sep=0, outer sep=0, custcol_0_0_0, anchor=base west] at  (437.0,438.0)  {Generator};}
\fill [line width=1.0pt, custcol_0_0_0]  (425.0,425.0) ellipse (2.0 and 2.0 );
\fill [line width=1.0pt, custcol_0_0_0]  (525.0,425.0) ellipse (2.0 and 2.0 );
\logisimfontB{\fontsize{12pt}{12pt}\bfseries\selectfont\node[inner sep=0, outer sep=0, custcol_0_0_0, anchor=base west] at  (721.0,393.0)  {ALUCtrl};}
\draw [line width=2.0pt, custcol_0_0_0 ]  (718.0,379.0) -- (735.0,379.0) -- (745.0,375.0) -- (755.0,379.0) -- (772.0,379.0) -- (772.0,398.0) -- (718.0,398.0) -- cycle;
\fill [line width=2.0pt, custcol_0_0_0]  (745.0,375.0) ellipse (2.0 and 2.0 );
\draw [line width=4.0pt, custcol_0_0_0 ]  (975.0,35.0) -- (975.0,115.0) -- (1015.0,95.0) -- (1015.0,55.0) -- cycle;
\logisimfontC{\fontsize{14pt}{14pt}\selectfont\node[inner sep=0, outer sep=0, custcol_87_87_87, anchor=base west] at  (982.0,60.0)  {0};}
\logisimfontB{\fontsize{12pt}{12pt}\bfseries\selectfont\node[inner sep=0, outer sep=0, custcol_0_0_0, anchor=base west] at  (981.0,78.0)  {Mux};}
\fill [line width=1.0pt, custcol_0_0_0]  (975.0,55.0) ellipse (2.0 and 2.0 );
\fill [line width=1.0pt, custcol_0_0_0]  (975.0,95.0) ellipse (2.0 and 2.0 );
\fill [line width=1.0pt, custcol_0_0_0]  (995.0,105.0) ellipse (2.0 and 2.0 );
\fill [line width=1.0pt, custcol_0_0_0]  (1015.0,75.0) ellipse (2.0 and 2.0 );
\draw [line width=3.0pt, custcol_0_0_0 ]  (595.0,95.0) -- (595.0,130.0) -- (605.0,135.0) -- (595.0,140.0) -- (595.0,175.0) -- (635.0,155.0) -- (635.0,115.0) -- cycle;
\logisimfontB{\fontsize{20pt}{20pt}\bfseries\selectfont\node[inner sep=0, outer sep=0, custcol_0_0_0, anchor=base west] at  (617.0,140.0)  {+};}
\fill [line width=1.0pt, custcol_0_0_0]  (595.0,115.0) ellipse (2.0 and 2.0 );
\fill [line width=1.0pt, custcol_0_0_0]  (595.0,155.0) ellipse (2.0 and 2.0 );
\fill [line width=1.0pt, custcol_0_0_0]  (635.0,135.0) ellipse (2.0 and 2.0 );
\logisimfontB{\fontsize{10pt}{10pt}\bfseries\selectfont\node[inner sep=0, outer sep=0, custcol_0_0_0, anchor=base west] at  (871.0,208.0)  {ReadMem};}
\draw [line width=2.0pt, custcol_0_0_0 ]  (868.0,195.0) -- (922.0,195.0) -- (922.0,212.0) -- (905.0,212.0) -- (895.0,216.0) -- (885.0,212.0) -- (868.0,212.0) -- cycle;
\fill [line width=2.0pt, custcol_0_0_0]  (895.0,215.0) ellipse (2.0 and 2.0 );
\draw [line width=2.0pt, custcol_0_0_0] (780.0,180.0) arc (90.0:-90.0:15.0 and 15.0 );
\draw [line width=2.0pt, custcol_0_0_0 ]  (780.0,150.0) -- (766.0,150.0) -- (766.0,180.0) -- (780.0,180.0) ;
\logisimfontB{\fontsize{10pt}{10pt}\bfseries\selectfont\node[inner sep=0, outer sep=0, custcol_0_0_0, anchor=base west] at  (804.0,149.0)  {Jump};}
\draw [line width=2.0pt, custcol_0_0_0 ]  (801.0,136.0) -- (832.0,136.0) -- (836.0,145.0) -- (832.0,154.0) -- (801.0,154.0) -- cycle;
\fill [line width=2.0pt, custcol_0_0_0]  (835.0,145.0) ellipse (2.0 and 2.0 );
\draw [line width=3.0pt, custcol_0_0_0 ]  (835.0,145.0) -- (855.0,145.0) -- (857.0,145.0) ;
\draw [line width=3.0pt, custcol_0_0_0 ]  (795.0,165.0) -- (855.0,165.0) -- (857.0,165.0) ;
\draw [line width=2.0pt, custcol_0_0_0 ]  (885.0,155.0) .. controls  (875.0,140.0)  ..  (855.0,140.0) .. controls  (863.0,155.0)  ..  (855.0,170.0) .. controls  (875.0,170.0)  ..  (885.0,155.0) -- cycle ;
\logisimfontB{\fontsize{10pt}{10pt}\bfseries\selectfont\node[inner sep=0, outer sep=0, custcol_0_0_0, anchor=base west] at  (706.0,159.0)  {Branch};}
\draw [line width=2.0pt, custcol_0_0_0 ]  (703.0,146.0) -- (742.0,146.0) -- (746.0,155.0) -- (742.0,164.0) -- (703.0,164.0) -- cycle;
\fill [line width=2.0pt, custcol_0_0_0]  (745.0,155.0) ellipse (2.0 and 2.0 );
\logisimfontB{\fontsize{20pt}{20pt}\bfseries\selectfont\node[inner sep=0, outer sep=0, custcol_0_0_0, anchor=base west] at  (825.0,353.0)  {Memory};}
\draw [line width=4.0pt, custcol_0_0_0 ]  (805.0,225.0) -- (924.0,225.0) ;
\draw [line width=4.0pt, custcol_0_0_0 ]  (925.0,225.0) -- (925.0,444.0) ;
\draw [line width=4.0pt, custcol_0_0_0 ]  (925.0,445.0) -- (806.0,445.0) ;
\draw [line width=4.0pt, custcol_0_0_0 ]  (805.0,445.0) -- (805.0,226.0) ;
\logisimfontB{\fontsize{20pt}{20pt}\bfseries\selectfont\node[inner sep=0, outer sep=0, custcol_0_0_0, anchor=base west] at  (842.0,334.0)  {Data};}
\draw [line width=1.0pt, custcol_0_0_0 ]  (891.0,444.0) -- (897.0,428.0) -- (904.0,445.0) ;
\logisimfontC{\fontsize{12pt}{12pt}\selectfont\node[inner sep=0, outer sep=0, custcol_87_87_87, anchor=base west] at  (814.0,269.0)  {Address};}
\logisimfontC{\fontsize{12pt}{12pt}\selectfont\node[inner sep=0, outer sep=0, custcol_87_87_87, anchor=base west] at  (813.0,409.0)  {Data};}
\fill [line width=1.0pt, custcol_0_0_0]  (805.0,265.0) ellipse (2.0 and 2.0 );
\fill [line width=1.0pt, custcol_0_0_0]  (805.0,405.0) ellipse (2.0 and 2.0 );
\fill [line width=1.0pt, custcol_0_0_0]  (835.0,225.0) ellipse (2.0 and 2.0 );
\fill [line width=1.0pt, custcol_0_0_0]  (895.0,225.0) ellipse (2.0 and 2.0 );
\fill [line width=1.0pt, custcol_0_0_0]  (925.0,335.0) ellipse (2.0 and 2.0 );
\draw [line width=4.0pt, custcol_0_0_0 ]  (675.0,185.0) -- (675.0,225.0) -- (715.0,265.0) -- (675.0,305.0) -- (675.0,344.0) -- (755.0,275.0) -- (755.0,255.0) -- cycle;
\logisimfontB{\fontsize{14pt}{14pt}\bfseries\selectfont\node[inner sep=0, outer sep=0, custcol_0_0_0, anchor=base west] at  (720.0,270.0)  {ALU};}
\logisimfontC{\fontsize{12pt}{12pt}\selectfont\node[inner sep=0, outer sep=0, custcol_87_87_87, anchor=base west] at  (723.0,256.0)  {zero};}
\fill [line width=1.0pt, custcol_0_0_0]  (675.0,205.0) ellipse (2.0 and 2.0 );
\fill [line width=1.0pt, custcol_0_0_0]  (675.0,325.0) ellipse (2.0 and 2.0 );
\fill [line width=1.0pt, custcol_0_0_0]  (745.0,245.0) ellipse (2.0 and 2.0 );
\fill [line width=1.0pt, custcol_0_0_0]  (745.0,285.0) ellipse (2.0 and 2.0 );
\fill [line width=1.0pt, custcol_0_0_0]  (755.0,265.0) ellipse (2.0 and 2.0 );
\end{tikzpicture}
}
    \vspace{-20pt}
    \caption{Esquemático do Uniciclo com ISA reduzida}
\end{figure}

\noindent Em que os túneis representam entradas e saídas do módulo de controle do processador, que foi abstraído na figura acima.

\begin{figure}[H]
    \centering
    % Important: If latex complains about unicode characters,
% please use "\usepackage[utf8x]{inputenc}" in your preamble
% You can change the size of the picture by putting it into the construct:
% 1) \resizebox{10cm}{!}{"below picture"} to scale horizontally to 10 cm
% 2) \resizebox{!}{15cm}{"below picture"} to scale vertically to 15 cm
% 3) \resizebox{10cm}{15cm}{"below picture"} a combination of above two
% It is not recomended to use the scale option of the tikzpicture environment.
\scalebox{0.5}{
\begin{tikzpicture}[x=1pt,y=-1pt,line cap=rect]
\def\logisimfontA#1{{\fontspec{CMU Serif}#1}} % cmr equivalent
\def\logisimfontB#1{{\fontspec{CMU Bright}#1}}
\definecolor{custcol_0_0_0}{RGB}{0, 0, 0}
\definecolor{custcol_ff_ff_ff}{RGB}{255, 255, 255}
\draw [line width=4.0pt, custcol_0_0_0 ]  (422.0,137.0) -- (422.0,157.0) ;
\draw [line width=3.0pt, custcol_0_0_0 ]  (262.0,27.0) -- (292.0,27.0) ;
\draw [line width=3.0pt, custcol_0_0_0 ]  (262.0,167.0) -- (292.0,167.0) ;
\draw [line width=3.0pt, custcol_0_0_0 ]  (262.0,87.0) -- (292.0,87.0) ;
\draw [line width=3.0pt, custcol_0_0_0 ]  (262.0,147.0) -- (292.0,147.0) ;
\draw [line width=3.0pt, custcol_0_0_0 ]  (262.0,67.0) -- (292.0,67.0) ;
\draw [line width=3.0pt, custcol_0_0_0 ]  (262.0,127.0) -- (292.0,127.0) ;
\draw [line width=3.0pt, custcol_0_0_0 ]  (262.0,47.0) -- (292.0,47.0) ;
\draw [line width=4.0pt, custcol_0_0_0 ]  (452.0,107.0) -- (482.0,107.0) ;
\draw [line width=4.0pt, custcol_0_0_0 ]  (262.0,107.0) -- (392.0,107.0) ;
\draw [line width=4.0pt, custcol_0_0_0 ]  (82.0,97.0) -- (152.0,97.0) ;
\begin{pgfpicture}
   \begin{pgfmagnify}{1pt}{-1pt}
      \pgfsetrectcap
      \pgfsetcornersarced{\pgfpoint{10.0}{10.0}}
      \pgfsetlinewidth{4.0}
      \color{custcol_0_0_0}
      \pgfsetfillopacity{1.0}
      \pgfpathrectanglecorners{\pgfpoint{152.0}{7.0}}{\pgfpoint{262.0}{187.0}}
      \pgfusepath{stroke}
   \end{pgfmagnify}
\end{pgfpicture}
\logisimfontB{\fontsize{20pt}{20pt}\fontseries{bx}\selectfont\node[inner sep=0, outer sep=0, custcol_0_0_0, anchor=base west] at  (171.0,103.0)  {Control};}
\fill [line width=1.0pt, custcol_0_0_0]  (152.0,97.0) ellipse (2.0 and 2.0 );
\fill [line width=1.0pt, custcol_0_0_0]  (262.0,27.0) ellipse (2.0 and 2.0 );
\fill [line width=1.0pt, custcol_0_0_0]  (262.0,47.0) ellipse (2.0 and 2.0 );
\fill [line width=1.0pt, custcol_0_0_0]  (262.0,67.0) ellipse (2.0 and 2.0 );
\fill [line width=1.0pt, custcol_0_0_0]  (262.0,87.0) ellipse (2.0 and 2.0 );
\fill [line width=1.0pt, custcol_0_0_0]  (262.0,107.0) ellipse (2.0 and 2.0 );
\fill [line width=1.0pt, custcol_0_0_0]  (262.0,127.0) ellipse (2.0 and 2.0 );
\fill [line width=1.0pt, custcol_0_0_0]  (262.0,147.0) ellipse (2.0 and 2.0 );
\fill [line width=1.0pt, custcol_0_0_0]  (262.0,167.0) ellipse (2.0 and 2.0 );
\logisimfontB{\fontsize{10pt}{10pt}\fontseries{bx}\selectfont\node[inner sep=0, outer sep=0, custcol_0_0_0, anchor=base west] at  (297.0,151.0)  {ALUSrc};}
\draw [line width=2.0pt, custcol_0_0_0 ]  (296.0,138.0) -- (337.0,138.0) -- (337.0,156.0) -- (296.0,156.0) -- (292.0,147.0) -- cycle;
\fill [line width=2.0pt, custcol_0_0_0]  (292.0,147.0) ellipse (2.0 and 2.0 );
\logisimfontB{\fontsize{10pt}{10pt}\fontseries{bx}\selectfont\node[inner sep=0, outer sep=0, custcol_0_0_0, anchor=base west] at  (297.0,71.0)  {Branch};}
\draw [line width=2.0pt, custcol_0_0_0 ]  (296.0,58.0) -- (334.0,58.0) -- (334.0,76.0) -- (296.0,76.0) -- (292.0,67.0) -- cycle;
\fill [line width=2.0pt, custcol_0_0_0]  (292.0,67.0) ellipse (2.0 and 2.0 );
\logisimfontB{\fontsize{10pt}{10pt}\fontseries{bx}\selectfont\node[inner sep=0, outer sep=0, custcol_0_0_0, anchor=base west] at  (297.0,131.0)  {WriteMem};}
\draw [line width=2.0pt, custcol_0_0_0 ]  (296.0,118.0) -- (349.0,118.0) -- (349.0,136.0) -- (296.0,136.0) -- (292.0,127.0) -- cycle;
\fill [line width=2.0pt, custcol_0_0_0]  (292.0,127.0) ellipse (2.0 and 2.0 );
\logisimfontB{\fontsize{10pt}{10pt}\fontseries{bx}\selectfont\node[inner sep=0, outer sep=0, custcol_0_0_0, anchor=base west] at  (297.0,91.0)  {Jump};}
\draw [line width=2.0pt, custcol_0_0_0 ]  (296.0,78.0) -- (326.0,78.0) -- (326.0,96.0) -- (296.0,96.0) -- (292.0,87.0) -- cycle;
\fill [line width=2.0pt, custcol_0_0_0]  (292.0,87.0) ellipse (2.0 and 2.0 );
\logisimfontB{\fontsize{10pt}{10pt}\fontseries{bx}\selectfont\node[inner sep=0, outer sep=0, custcol_0_0_0, anchor=base west] at  (297.0,171.0)  {WriteReg};}
\draw [line width=2.0pt, custcol_0_0_0 ]  (296.0,158.0) -- (344.0,158.0) -- (344.0,176.0) -- (296.0,176.0) -- (292.0,167.0) -- cycle;
\fill [line width=2.0pt, custcol_0_0_0]  (292.0,167.0) ellipse (2.0 and 2.0 );
\logisimfontB{\fontsize{10pt}{10pt}\fontseries{bx}\selectfont\node[inner sep=0, outer sep=0, custcol_0_0_0, anchor=base west] at  (297.0,31.0)  {Mem2Reg};}
\draw [line width=2.0pt, custcol_0_0_0 ]  (296.0,18.0) -- (348.0,18.0) -- (348.0,36.0) -- (296.0,36.0) -- (292.0,27.0) -- cycle;
\fill [line width=2.0pt, custcol_0_0_0]  (292.0,27.0) ellipse (2.0 and 2.0 );
\logisimfontB{\fontsize{10pt}{10pt}\fontseries{bx}\selectfont\node[inner sep=0, outer sep=0, custcol_0_0_0, anchor=base west] at  (297.0,51.0)  {ReadMem};}
\draw [line width=2.0pt, custcol_0_0_0 ]  (296.0,38.0) -- (348.0,38.0) -- (348.0,56.0) -- (296.0,56.0) -- (292.0,47.0) -- cycle;
\fill [line width=2.0pt, custcol_0_0_0]  (292.0,47.0) ellipse (2.0 and 2.0 );
\logisimfontB{\fontsize{16pt}{16pt}\fontseries{bx}\selectfont\node[inner sep=0, outer sep=0, custcol_0_0_0, anchor=base west] at  (4.0,102.0)  {OPCODE};}
\draw [line width=2.0pt, custcol_0_0_0 ]  (0.0,85.0) -- (79.0,85.0) -- (79.0,87.0) -- (83.0,97.0) -- (79.0,107.0) -- (79.0,109.0) -- (0.0,109.0) -- cycle;
\fill [line width=2.0pt, custcol_0_0_0]  (82.0,97.0) ellipse (2.0 and 2.0 );
\draw [line width=3.0pt, custcol_0_0_0]  (422.0,107.0) ellipse (30.0 and 30.0 );
\logisimfontB{\fontsize{12pt}{12pt}\fontseries{bx}\selectfont\node[inner sep=0, outer sep=0, custcol_0_0_0, anchor=base west] at  (409.0,108.0)  {ALU};}
\logisimfontB{\fontsize{12pt}{12pt}\fontseries{bx}\selectfont\node[inner sep=0, outer sep=0, custcol_0_0_0, anchor=base west] at  (399.0,117.0)  {Control};}
\fill [line width=1.0pt, custcol_0_0_0]  (392.0,107.0) ellipse (2.0 and 2.0 );
\fill [line width=1.0pt, custcol_0_0_0]  (422.0,137.0) ellipse (2.0 and 2.0 );
\fill [line width=1.0pt, custcol_0_0_0]  (452.0,107.0) ellipse (2.0 and 2.0 );
\logisimfontB{\fontsize{12pt}{12pt}\fontseries{bx}\selectfont\node[inner sep=0, outer sep=0, custcol_0_0_0, anchor=base west] at  (406.0,173.0)  {Funct};}
\draw [line width=2.0pt, custcol_0_0_0 ]  (403.0,161.0) -- (412.0,161.0) -- (422.0,157.0) -- (432.0,161.0) -- (442.0,161.0) -- (442.0,180.0) -- (403.0,180.0) -- cycle;
\fill [line width=2.0pt, custcol_0_0_0]  (422.0,157.0) ellipse (2.0 and 2.0 );
\logisimfontB{\fontsize{12pt}{12pt}\fontseries{bx}\selectfont\node[inner sep=0, outer sep=0, custcol_0_0_0, anchor=base west] at  (487.0,111.0)  {ALUCtrl};}
\draw [line width=2.0pt, custcol_0_0_0 ]  (486.0,97.0) -- (538.0,97.0) -- (538.0,118.0) -- (486.0,118.0) -- (486.0,117.0) -- (482.0,107.0) -- cycle;
\fill [line width=2.0pt, custcol_0_0_0]  (482.0,107.0) ellipse (2.0 and 2.0 );
\end{tikzpicture}
}
    \caption{Módulo de controle do Uniciclo com ISA reduzida}
\end{figure}

\subsubsection{Tarefa I}
\begin{tcolorbox}[title=Enunciado, colback=blue!5!white, colframe=blue!75!black]
Analise o programa \cod{de1.s} que testa a corretude da implementação de todas as $9 + 3$ instruções e
teste no RARS. \textit{Dica: O registrador \cod{t0} é usado para visualizar resultados!}
\end{tcolorbox}
O programa \cod{de1.s}, que testa a corretude da implementação das instruções do RISC-V no RARS, ou seja, verifica se as instruções são executadas corretamente, foi testado utilizando a execução passo a passo, e os resultados foram visualizados por meio do registrador \cod{t0}, de acordo com o que foi sugerido. Os registradores \cod{t1} e \cod{t2} são os registradores temporários utilizados para as operações e \cod{gp} é o ponteiro para a área de dados, inicializado em \cod{0x100100000}.

O programa testa as instruções \cod{add}, \cod{sub}, \cod{and}, \cod{or}, \cod{slt}, \cod{lw},
\cod{sw}, \cod{beq}, \cod{jal}, \cod{jalr}, \cod{addi} e \cod{lui}.

\begin{table}[H]
    \centering
    \footnotesize
    \resizebox{\textwidth}{!}{
    \renewcommand{\arraystretch}{1.3}
    \setlength{\tabcolsep}{4pt}
    \begin{tabular}{|c|c|c|c|}
        \hline
        \rowcolor{black}
        \textcolor{white}{\textbf{Instrução}} &
        \textcolor{white}{\textbf{Código}} &
        \textcolor{white}{\textbf{Entrada}} &
        \textcolor{white}{\textbf{Saída}} \\ \hline
        
        lui & \cod{lui gp,0x10010} & \cod{gp}, \cod{imm = 0x10010} & \cod{gp = 0x10010000} \\ \hline
        \rowcolor{gray!20}
        lw & \cod{lw t1,0(gp)} & \cod{gp = 0x10010000}, \cod{mem[0x10010000] = 0xFFFFFF0F} & \cod{t1 = 0xFFFFFF0F} \\ \hline
        addi & \cod{addi t2,zero,0x777} & \cod{zero = 0x00000000}, \cod{imm = 0x777} & \cod{t2 = 0x00000777} \\ \hline
        \rowcolor{gray!20}
        and & \cod{and t0,t1,t2} & \cod{t1 = 0xFFFFFF0F}, \cod{t2 = 0x00000777} & \cod{t0 = 0x00000707} \\ \hline
        or & \cod{or t0,t1,t2} & \cod{t1 = 0xFFFFFF0F}, \cod{t2 = 0x00000777} & \cod{t0 = 0xFFFFFF7F} \\ \hline
        \rowcolor{gray!20}
        add & \cod{add t0,t2,t1} & \cod{t2 = 0x00000777}, \cod{t1 = 0xFFFFFF0F} & \cod{t0 = 0x00000686} \\ \hline
        sub & \cod{sub t0,t2,t1} & \cod{t2 = 0x00000777}, \cod{t1 = 0xFFFFFF0F} & \cod{t0 = 0x00000868} \\ \hline
        \rowcolor{gray!20}
        slt & \cod{slt t0,t1,t2} & \cod{t1 = 0xFFFFFF0F}, \cod{t2 = 0x00000777} & \cod{t0 = 0x00000001} \\ \hline
        slt & \cod{slt t0,t2,t1} & \cod{t2 = 0x00000777}, \cod{t1 = 0xFFFFFF0F} & \cod{t0 = 0x00000000} \\ \hline
        \rowcolor{gray!20}
        beq & \cod{beq t0,zero,PULA} & \cod{t0 = 0x00000000}, \cod{zero = 0x00000000} & Salta para \cod{PULA} \\ \hline
        jal & \cod{jal PROC} & \cod{PC atual}, \cod{ra = PC + 4} & Chama \cod{PROC}, \cod{ra = end. retorno} \\ \hline
        \rowcolor{gray!20}
        sw & \cod{sw t0,4(gp)} & \cod{t0 = 0x0000007F}, \cod{gp = 0x10010000} & \cod{mem[0x10010004] = 0x0000007F} \\ \hline
    \end{tabular}
    }
    \caption{Código e saídas associadas}
\end{table}

Uma explicação resumida do código: Inicializa-se o ponteiro de memória carregando o endereço \cod{0x10010000} no registrador \cod{gp}, e carrega-se o valor \cod{0xFFFFFF0F} da memória para \cod{t1} e inicializa-se \cod{t2} com \cod{0x777}. Então, é executado várias operações lógicas e aritméticas com \cod{t1} e \cod{t2}, AND e OR, soma (\cod{add} e \cod{addi}) e subtração (\cod{sub}), comparações (\cod{slt}) que verificam se um valor é menor que o outro, teste desvio condicional e chamada de subrotina.


\subsubsection{Tarefa II}
\begin{tcolorbox}[title=Enunciado, colback=blue!5!white, colframe=blue!75!black]
Implemente o Banco de Registradores com 3 leituras simultâneas: \cod{rs1}, \cod{rs2} e \cod{disp}.

\centering \footnotesize Stack Pointer (\cod{sp}) inicial: \cod{0x1001\_03FC} \hspace{1cm} Global Pointer (\cod{gp}) inicial: \cod{0x1001\_0000}
\end{tcolorbox}
O Banco de Registradores é um componente fundamental e funciona como a memória mais rápida e acessível do processador. O programa com 3 leituras simultâneas, \cod{rs1}, \cod{rs2} e \cod{disp}, já estava quase todo implementado em VHDL (xreg.vhd) - faltava apenas atribuir ao \cod{oREGD} (o \cod{disp} do enunciado) o valor no registrador da posição \cod{iDisp} do banco de registradores.

O banco de registradores é iniciado (e reinciado, quando a entrada \cod{iRST} está habilitada) com todos os registradores zerados, com exceção do \cod{sp} e do \cod{gp}, que têm os valores definidos no enunciado. Esses valores representam, respectivamente, o endereço da pilha de memória e a área de variáveis globais. Essas definições foram feitas no arquivo riscv\_pkg.vhd.

A operação ocorre em dois modos: atualização \textbf{assíncrona} dos registradores de saída \cod{oREGA}, \cod{oREGB} e \cod{oRegD} nos endereços \cod{iRS1}, \cod{iRS2} e \cod{iDISP}, respectivamente; e escrita \textbf{síncrona} no banco de registradores (a não ser que o \textit{reset} esteja habilitado, nesse caso, são escritos os valores do estado inicial nos registradores de forma \textbf{assíncrona}). Quando a CPU precisa guardar o resultado de algum cálculo ou armazenar algum valor da memória, o sinal \cod{iWREN} é ativado e, no endereço do registrador destino \cod{iRD}, é escrito o dado em \cod{iDATA}.

\subsubsection{Tarefa III}
\begin{tcolorbox}[title=Enunciado, colback=blue!5!white, colframe=blue!75!black]
Implemente o Gerador de Imediatos. 
\end{tcolorbox}
O Gerador de Imediatos, também já quase todo implementado (gen\_imm.vhd), recebe 32 bits da instrução e gera um imediato de 32 bits. Este componente é responsável por extrair e processar os valores imediatos presentes nas instruções, considerando que os campos de imediatos não possuem um padrão uniforme. A implementação consiste em um case, que é sintetizado em um multiplexador, em que dependendo do opcode (o tipo da instrução (I, S, B, U, J)) geramos imediatos com procedimentos distintos.

Inicialmente, é necessário analisar os 7 bits menos significativos da instrução, que são aqueles que correspondem ao campo opcode e vão servir como um seletor para determinar qual o formato de instrução que está sendo processado.

O primeiro caso, é das instruções Tipo-I, que, no nosso projeto, agrupam os opcodes \cod{OPC\_LOAD}, \cod{OPC\_OPIMM}, \cod{OPC\_JALR}. Nesse caso, o imediato tem 12 bits (os bits 31 a 20 da instrução), precisamos apenas extendê-lo para 32 bits.

O segundo caso tratato são as instruções Tipo-S (\textit{store}). Nelas o imediato é dividido em duas partes não contíguas na instrução, os bits 31 a 25 são os mais significativos, e os bits 11 a 7 são os menos significativos, gerando-se assim os 12 bits. O resultado da concatenação é estendido para 32 bits.

O terceito caso consiste nas instruções Tipo-B (\cod{branch}), que possuem uma estrutura mais complexa que as anteriores. Seus campos estão espalhados em quatro posições diferentes na instrução, do mais ao menos significativo: o bit 31, bit 7, bits 30 a 25 e 11 a 8. Eles são concatenados nessa sequência e, então, é adicionado um `0' no bit menos significativo, já que, pela forma como é organizada a memória, as instruções ficam em endereços pares. O resultado dessa sequência de manipulações é um sinal com 13 bits, que precisa ser estendido para 32.

O quarto caso, as instruções Tipo-J (\cod{jal}), é similar ao anterior: os bits estão espalhados e são reordenados, além de ser concatenado à direita o bit `0'. A diferença é que as instruções de Tipo-J dedicam 20 bits ao imediato, de forma que, com a concatenação com 0, já são 21 bits, e o resultado só precisa ser estendido em 11 bits.

O quinto e último caso de interesse (pois, nos restantes, o imediato é um conjunto de \textit{don't cares} que arbitramos como 32 zeros) é o das instruções Tipo-U. Nesse caso, na instrução há 20 bits contíguos reservados ao imediado; basta estender essa sequência para 32 bits e pronto. Esse caso não estava tratada no esqueleto de projeto fornecido.

\subsubsection{Tarefa V}
\begin{tcolorbox}[title=Enunciado, colback=blue!5!white, colframe=blue!75!black]
Implemente a ULA mínima necessária (\cod{add}, \cod{sub}, \cod{and}, \cod{or}, \cod{slt}, \cod{zero}).
\end{tcolorbox}
A ULA (Unidade Lógica e Aritmética) é um componente fundamental responsável por realizar operações tanto lógicas quanto aritméticas; sendo elas, no caso deste laboratório: adição, subtração, \textit{bitwise AND}, \textit{bitwise OR} e SLT (Set if Less Than). Adicionalmente, ela deve ser capaz de fazer o `OU' entre todos os bits de uma palavra, pois isso é necessário para calcular a saída `zero'.

Na implementação desse módulo, seguimos um estilo híbrido entre estrutural e comportamental: ao mesmo tempo que instanciamos um somador para realizar somas e subtrações, utilizamos abstrações de alto nível, nesse caso, as estruturas \cod{case} e \cod{when ... else}, para implementar o \cod{slt}, \cod{or}, \cod{and} e calcular a saída \cod{zero}.

A ULA implementada recebe como entrada os valores dos dois operandos e um sinal de 4 bits \cod{ALUCtrl}, que indica qual das operações previamente citadas deve ser realizada. Dessa forma, a implementação é dada a partir de um `case' no qual, para cada código recebido pelo sinal ``ALUCtrl'', é realizada uma operação lógica ou aritmética.

Por último, a ULA tem dois sinais de saída: um para o resultado da operação realizada e outro para o sinal \cod{zero}, que é ligado caso o resultado da operação seja `0' - ele é usado especificamente no caso do \cod{branch}, se a subtração dos operandos for `0', o `zero' é ligado o posterior AND desse com sinal com o sinal de controle `branch' faz com que o endereço do salto seja escrito no PC.


\subsubsection{Tarefa VI}
\begin{tcolorbox}[title=Enunciado, colback=blue!5!white, colframe=blue!75!black]
Implemente o Controlador da ULA e o Bloco Controlador.
\end{tcolorbox}
O Bloco Controlador é o verdadeiro ``maestro'' do processador, ou seja, ele é responsável por gerar os sinais de controle necessários para gerenciar todo o caminho de dados e para coordenar quase todos os outros componentes da CPU, entre eles: a memória de dados, o banco de registradores, o controlador da ULA e vários multiplexadores.

Esse módulo recebe como entrada o `opcode' (código de operação), um sinal de 7 bits que determina qual é a operação básica que implementa a instrução. Considerando a ISA reduzida, são possíveis oito `opcodes': um para todas as instruções Tipo-R e mais uma para cada uma das outras instruções (\cod{sw}, \cod{lw}, \cod{addi}, \cod{beq}, \cod{jalr}, \cod{jal} e \cod{lui}).

Além disso, o Bloco de Controle, envia, como saída, os sinais que coordenam os outros componentes do processador: para cada `opcode', o Bloco de Controle determina esses sinais para controlar exatamente como o resto do processador deve agir para que a instrução recebida seja, de fato, executada. Seguem os sinais de saída:

\begin{itemize}
    \item \cod{mem2Reg} - 1 bit - Indica qual sinal deve estar disponível para ser escrito no banco de registradores. Caso `0': resultado da ULA. Caso `1': saída da memória de dados;
    \item \cod{memRead} - 1 bit - Determina se há necessidade de leitura na memória de dados. Caso `0': desabilita a leitura. Caso `1': habilita a leitura;
    \item \cod{branch} - 1 bit - Determina se PC deve receber PC + 4 ou PC + imediato. Caso `0': PC + 4. Caso `1': PC + Imm;
    \item \cod{jump} - 1 bit - Determina se haverá salto incondicional. Caso `0': desabilita o salto. Caso `1': habilita o salto;
    \item \cod{ALUOp} - 2 bits - Determina a entrada do Controlador da ULA;
    \item \cod{memWrite} - 1 bit - Determina se a escrita na memória de dados deve estar habilitada. Caso `0': desabilita. Caso `1': habilita;
    \item \cod{ALUSrc} - 1 bit - Indica a origem do segundo operando da ULA. Caso `0': sinal com origem no banco de registradores. Caso `1': sinal com origem no gerador de imediatos;
    \item \cod{regWrite} - 1 bit - Determina se a escrita no banco de registradores deve estar habilitada. Caso `0': desabilita. Caso `1': habilita.
\end{itemize}

Uma observação importante é que o sinal \cod{jump} não existe no esquemático fornecido no roteiro e tampouco era estritamente necessário para o projeto. Poderíamos ter implementado instruções de salto incondicional forçando os sinais \cod{branch} e \cod{zero} iguais a `1'. Nesse caso, seria necessário criar um \cod{ALUOp} para \cod{jal} e \cod{jalr} que fizesse \cod{zero <= 1}. Essa solução, apesar de economizar um sinal, foi considerada \textit{ad-hoc} e prejudicial ao significado dos sinais de controle.

A implementação do Bloco de Controle relaciona sua entrada com suas saídas de acordo com a tabela a seguir:

\begin{table}[H]
    \centering
    \footnotesize
    \renewcommand{\arraystretch}{1.3}
    \setlength{\tabcolsep}{6pt}
    \resizebox{\textwidth}{!}{%
    \begin{tabular}{|c|c|c|c|c|c|c|c|c|}
        \hline
        \rowcolor{black}
        % ==== Cabeçalho com célula diagonal preta ====
        \diagbox[linecolor=white, width=2.5cm,height=2\baselineskip]{\textcolor{white}{\textbf{Opcode}}}{\textcolor{white}{\textbf{Saída}}} &
        \textcolor{white}{\textbf{mem2Reg}} &
        \textcolor{white}{\textbf{memRead}} &
        \textcolor{white}{\textbf{branch}} &
        \textcolor{white}{\textbf{jump}} &
        \textcolor{white}{\textbf{ALUOp}} &
        \textcolor{white}{\textbf{memWrite}} &
        \textcolor{white}{\textbf{ALUSrc}} &
        \textcolor{white}{\textbf{regWrite}} \\ \hline

        % ==== Linhas da tabela ====
        \cod{OPC\_RTYPE} & 0 & 0 & 0 & 0 & \cod{10} & 0 & 0 & 1 \\ \hline
        \rowcolor{gray!20}
        \cod{OPC\_OPIMM} & 0 & 0 & 0 & 0 & \cod{10} & 0 & 1 & 1 \\ \hline
        \cod{OPC\_STORE} & X & 0 & 0 & 0 & \cod{00} & 1 & 1 & 0 \\ \hline
        \rowcolor{gray!20}
        \cod{OPC\_LOAD} & 1 & 1 & 0 & 0 & \cod{00} & 0 & 1 & 1 \\ \hline
        \cod{OPC\_BRANCH} & X & 0 & 1 & 0 & \cod{01} & 0 & 0 & 0 \\ \hline
        \rowcolor{gray!20}
        \cod{OPC\_JAL} & X & 0 & 0 & 1 & XX & 0 & X & 1 \\ \hline
        \cod{OPC\_JALR} & 0 & 0 & 0 & 1 & \cod{00} & 0 & 1 & 1 \\ \hline
        \rowcolor{gray!20}
        \cod{OPC\_LUI} & 0 & 0 & 0 & 0 & \cod{00} & 0 & 1 & 1 \\ \hline
    \end{tabular}
    }
    \caption{Sinais de controle gerados para cada opcode}
    \label{tab:opcode_controle}
\end{table}

Perceba que, até agora, o módulo de controle, chamado no projeto de \cod{general_control}, só recebeu o \cod{opcode}. Então para que servem \cod{funct7} e \cod{funct3}? Na verdade, o módulo de controle não está completo; falta o controle da ULA. Este é responsável por definir qual operação a ULA realizará - note que o ALUOp não é suficiente para isso, dado, por exemplo, que ele é o mesmo para todas as operações Tipo-R, como \cod{add} e \cod{and}, que claramente não realizam a mesma operação na ULA; ele recebe o \cod{ALUOp}, o \cod{funct7} e o \cod{funct3} (na nossa implementação, o \cod{funct10}). Essa separação se deve ao modelo hierárquico do módulo de controle seguido na nosssa implementação.

Em resumo, o controlador da ULA (chamado de \cod{alu_control} no projeto), determina, através do seu sinal de saída \cod{ALUCtrl}, exatamente qual operação lógico-aritmética deve ser executada.


Acesso à memória (lw, sw) - “00” - [add]
Desvio (branch) - “01” - [sub]
Lógico-aritméticas - “10” - [add, sub, and, or, slt]
	Então, caso o sinal `ALUOp' seja “00”, o Controlador da ULA determina para a ULA que a operação realizada deverá ser uma soma, da mesma forma que, caso o sinal `ALUOp' seja “01”, será determinada uma subtração e, por último, caso seja “10”, a partir dos sinais `funct7' e `funct3', será determinada a operação a ser realizada.




\section{Conclusão}

\end{document}