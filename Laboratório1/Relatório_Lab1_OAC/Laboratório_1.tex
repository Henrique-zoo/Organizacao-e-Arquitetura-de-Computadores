\documentclass[a4paper,12pt]{article}
\usepackage{float}
%\usepackage[table]{xcolor} % Para colorir as linhas da tabela
%\usepackage{array} % Para melhorar a formatação de tabelas
\usepackage[brazil]{babel}
\usepackage{cite}
\usepackage[utf8]{inputenc}
\usepackage{amsmath}
\usepackage{amssymb}
\usepackage{multicol}
\usepackage{graphicx}
%\usepackage{animate}
%\usepackage{tabularx}
\usepackage[font=footnotesize, labelfont=bf, listformat=empty]{caption}
\usepackage{geometry}
\usepackage{listings}
\usepackage{lastpage}
%\usepackage{subcaption}
\usepackage[skins,xparse,breakable]{tcolorbox}
%\usepackage{longtable}
\usepackage{minted}
\usepackage{titlesec}
\usepackage{fancyhdr}
\usepackage{setspace}
\usepackage{booktabs}
\usepackage{enumitem}
\usepackage[colorlinks=true, allcolors=black]{hyperref}
\usepackage{gnuplot-lua-tikz}


%\usetikzlibrary{arrows,shapes,automata,positioning,calc}
% Configurações de margens e espaçamento
\geometry{a4paper, left=2.5cm, right=2.5cm, top=2.5cm, bottom=2cm}
\onehalfspacing

% Configurações de títulos
\titleformat{\section}{\normalfont\Large\bfseries}{\thesection}{1em}{}
\titleformat{\subsection}{\normalfont\large\bfseries}{\thesubsection}{1em}{}
\titleformat{\subsubsection}
  {\normalfont\large}
  {\textbf{\thesubsubsection}}
  {0.25em}{}

\setcounter{tocdepth}{3}
\newcommand{\cod}[1]{\texttt{#1}}

\renewcommand{\listingscaption}{Código}
\newenvironment{code}{\captionsetup{type=listing}}{}

% Configurações de cores
\definecolor{LightSeaGreen}{rgb}{0.1255, 0.6980, 0.6667}
\definecolor{whitesmoke}{rgb}{0.96, 0.96, 0.96}
\definecolor{cinza}{RGB}{160, 160, 160}
\definecolor{cinza_claro}{RGB}{240, 240, 240} % Cinza claro

\newtcolorbox{codebox}[1]{
    title=#1,
    fonttitle=\small\bfseries,
    colback=blue!2.5!white,
    colframe=blue!62!white,
    arc=2pt,
    boxrule=0.8pt,
    left=4pt,
    right=4pt,
    top=4pt,
    bottom=4pt,
    breakable,
    enhanced jigsaw
}
% Configurações do minted para VHDL
\setminted[asm]{
    fontsize=\scriptsize,
    style=manni,
    linenos=true,
    breaklines=true,
    mathescape=true,
    bgcolor=cinza_claro,
    breakanywhere=true,
}

\setminted[fortran]{
    fontsize=\scriptsize,
    style=emacs,
    linenos=true,
    breaklines=true,
    mathescape=true,
    bgcolor=cinza_claro,
    breakanywhere=true,
}

\newcommand{\hex}[1]{\textcolor{green!50!black}{#1}}

% Configurações de cabeçalho e rodapé
\pagestyle{fancy}
\fancyhf{}
\fancyhead[L]{\footnotesize{Relatório do Laboratório 1}}
\fancyhead[R]{\footnotesize{2025/2 - Turma 02}}
\fancyfoot[R]{\footnotesize{Página \hspace{0.05cm} \thepage \hspace{0.05cm} de \pageref{LastPage}}}
\fancyfoot[L]{\footnotesize{Organização e Arquitetura de Computadores - CIC0099}}

% Adiciona uma linha preta acima do rodapé
\renewcommand{\footrulewidth}{0.4pt} % Espessura da linha
\renewcommand{\footrule}{\vbox to 0pt{\hrule width \textwidth height \footrulewidth \vss}} % Desenha a linha

% Novo Comando para chamar a capa
\newcommand{\capa}{
    \begin{titlepage}
        \begin{multicols}{2}
            \begin{flushleft}
                \includegraphics[width=0.45\linewidth]{Recursos/Imagens/UnB_logo.png}
            \end{flushleft}
            \columnbreak
            \begin{flushright}
                Universidade de Brasília \\
                Instituto de Exatas \\
                Departamento de Ciência da Computação
            \end{flushright}
        \end{multicols}
        \begin{center}
        \vspace{-20pt}
        \rule{\textwidth}{0.4pt}
        \end{center}
        \vspace{0.6cm}
        \begin{center}
            {\Huge \textbf{Relatório do Laboratório 1}} \\[0.5em]
            {\Huge \textbf{Assembly RISC-V}} \\ [0.5em]
            {\large CIC0099 - Organização e Arquitetura de Computadores} \\[0.25em]
            {\large Turma 02 - Prof. Dr. Ricardo Pezzuol Jacobi} \\
            \vfill
            {\Large \textbf{Grupo B4}} \\
            \begin{table}[H]
                \centering
                \begin{tabular}{l r}
                    Henrique Morcelles Salum & 232003008 \\
                    Athos Calixto Muniz & 211068261 \\
                    Gabriela Fernanda Rodrigues Costa & 180120859 \\
                    Cauê Araújo Euzebio & 211028195 \\
                    Lucas Silva Nóbrega & 180035096 \\
                \end{tabular}
                \label{tab:placeholder}
            \end{table}
        \end{center}
    \end{titlepage}
}

\begin{document}

\capa

% Sumário
\newpage
\tableofcontents
\newpage

\section{Introdução}
O laboratório 1 de OAC é separado em três grandes experimentos - esses, por sua vez, subdivididos em diversas tarefas - e visa familiarizar o aluno ao Assembly RISC-V e ao \textit{RISC-V Assembler and Runtime Simulator} (RARS). Notadamente, o primeiro experimento concentra-se no RARS; os outros dois, no RISC-V.

Mais detalhadamente, no primeiro experimento, analisa-se o desempenho de um código de ordenação fornecido; o segundo experimento requer o uso de um compilador cruzado de C para RISC-V, com o objetivo de gerar e corrigir códigos em Assembly; o terceiro experimento propõe o desenvolvimento de um algoritmo para a transformada discreta de Fourier em RISC-V.

\section{Métodos e Análise}

\subsection{Primeiro Experimento}
É fornecido o programa, em Assembly, \cod{sort.s}, de ordenamento de um vetor. Considere um processador RISC-V com frequência de clock 50MHz e CPI=1 e os vetores de entrada de $n$ elementos $V_0\left[n\right] = \{1, \ 2, \ \dots, \ n\}$, já ordenado, e $V_1\left[n\right] = \{n, \ n-1, \ \dots, \ 1 \}$, inversamente ordenado.

\subsubsection{Tarefa I}
\begin{tcolorbox}[title=Enunciado, colback=blue!5!white, colframe=blue!75!black]
Para o procedimento \cod{sort}, escreva as equações dos tempos de execução, $t_0[n]$ e $t_1[n]$, em função de $n$.
\end{tcolorbox}

Primeiro, devemos lembrar que o tempo de execução pode ser calculado pela seguinte equação:
\begin{equation}
    t_{exec} =\frac{I \cdot CPI}{f_{clock}} \label{eq:tempo}
\end{equation}
Note que já temos $CPI$ e $f_{clock}$, só resta descobrir $I$ (a quantidade de instruções executadas).

O RARS pode ser utilizado para esse fim: no menu, escolhemos Tools $>$ Instruction Counter, conectamos essa ferramenta ao código, montamos, adicionamos pontos de quebra na chamada do subprocedimento \cod{SORT} e na linha seguinte e executamos o programa. Realizando esse passo a passo para vários vetores ordenados e inversamente ordenados de diferentes tamanhos $n$, observou-se que
\[
    I_0[n] = \begin{cases}
        18, &\text{se } n = 0 \\
        10n + 13, &\text{se } n \geq 1
    \end{cases}
\]
e
\[
    I_1[n] = 9n^2 - 4n + 18, \text{ se } n \geq 0
\]
De acordo com a Equação \eqref{eq:tempo}, temos que
\begin{equation}
    t_0[n] = \begin{cases}
        0,36, &\text{se } n = 0 \\
        0,2n + 0,26, &\text{se } n \geq 1
    \end{cases}
    \quad \left[ \mu s\right] \label{eq:t0}
\end{equation}
e
\begin{equation}
    t_1[n] = 0,18 n^2 - 0,08n + 0,36, \text{ se } n \geq 0 \quad \left[ \mu s\right] \label{eq:t1}
\end{equation}

\subsubsection{Tarefa II}
\begin{tcolorbox}[title=Enunciado, colback=blue!5!white, colframe=blue!75!black]
Para $n = \left\{10,\ 20,\ 30,\ 40,\ 50,\ 60,\ 70,\ 80,\ 90,\ 100 \right\}$, plote (em escala!) as duas curvas, $t_0[n]$ e $t_1[n]$, em um mesmo gráfico $n \times t$.
\end{tcolorbox}

Utilizando as Equações \eqref{eq:t0} e \eqref{eq:t1}, fizemos o programa a seguir, em Fortran, que calcula $t_0[n]$ e $t_1[n]$ para os $n$ de interesse e escreve esses dados em um arquivo.
\begin{code}
    \begin{minted}{fortran}
program exec_time
    implicit none
    integer :: iunit, i
    
    open(newunit=iunit,file='oac_t0.dat',status='replace')
    do i = 1, 10
        write(iunit, *) i*10, t0(i*10)
    end do
    close(iunit)
    
    open(newunit=iunit,file='oac_t1.dat',status='replace')
    do i = 1, 10
        write(iunit, *) i*10, t1(i*10)
    end do
    close(iunit)
contains
    double precision function t0(n)
        integer, intent(in) :: n
        t0 = 0.26d0 + 0.2d0 * n
    end function
    double precision function t1(n)
        integer, intent(in) :: n
        t1 = 0.18d0*n*n - 0.08d0*n + 0.36d0
    end function
end program exec_time
    \end{minted}
    \vspace{-12pt}
    \caption{Programa em Fortran para calcular $t_0[n]$ e $t_1[n]$ para os argumentos pedidos}
\end{code}
\noindent A partir dos dados obtidos, plotamos o gráfico a seguir com gnuplot.

\begin{figure}[H]
    \centering
    \scalebox{1}{
\begin{tikzpicture}[gnuplot]
%% generated with GNUPLOT 6.0p0 (Lua 5.4; terminal rev. Jun 2020, script rev. 118)
%% qui 09 out 2025 15:21:36
\path (0.000,0.000) rectangle (12.500,8);
\gpcolor{color=gp lt color axes}
\gpsetlinetype{gp lt axes}
\gpsetdashtype{gp dt axes}
\gpsetlinewidth{0.50}
\draw[gp path] (1.504,1.024)--(11.947,1.024);
\gpcolor{color=gp lt color border}
\gpsetlinetype{gp lt border}
\gpsetdashtype{gp dt solid}
\gpsetlinewidth{1.00}
\draw[gp path] (1.504,1.024)--(1.684,1.024);
\draw[gp path] (11.947,1.024)--(11.767,1.024);
\node[gp node right] at (1.5,1.024) {\scriptsize $0$};
\gpcolor{color=gp lt color axes}
\gpsetlinetype{gp lt axes}
\gpsetdashtype{gp dt axes}
\gpsetlinewidth{0.50}
\draw[gp path] (1.504,1.805)--(11.947,1.805);
\gpcolor{color=gp lt color border}
\gpsetlinetype{gp lt border}
\gpsetdashtype{gp dt solid}
\gpsetlinewidth{1.00}
\draw[gp path] (1.504,1.805)--(1.684,1.805);
\draw[gp path] (11.947,1.805)--(11.767,1.805);
\node[gp node right] at (1.5,1.805) {\scriptsize $200$};
\gpcolor{color=gp lt color axes}
\gpsetlinetype{gp lt axes}
\gpsetdashtype{gp dt axes}
\gpsetlinewidth{0.50}
\draw[gp path] (1.504,2.586)--(11.947,2.586);
\gpcolor{color=gp lt color border}
\gpsetlinetype{gp lt border}
\gpsetdashtype{gp dt solid}
\gpsetlinewidth{1.00}
\draw[gp path] (1.504,2.586)--(1.684,2.586);
\draw[gp path] (11.947,2.586)--(11.767,2.586);
\node[gp node right] at (1.5,2.586) {\scriptsize $400$};
\gpcolor{color=gp lt color axes}
\gpsetlinetype{gp lt axes}
\gpsetdashtype{gp dt axes}
\gpsetlinewidth{0.50}
\draw[gp path] (1.504,3.366)--(11.947,3.366);
\gpcolor{color=gp lt color border}
\gpsetlinetype{gp lt border}
\gpsetdashtype{gp dt solid}
\gpsetlinewidth{1.00}
\draw[gp path] (1.504,3.366)--(1.684,3.366);
\draw[gp path] (11.947,3.366)--(11.767,3.366);
\node[gp node right] at (1.5,3.366) {\scriptsize $600$};
\gpcolor{color=gp lt color axes}
\gpsetlinetype{gp lt axes}
\gpsetdashtype{gp dt axes}
\gpsetlinewidth{0.50}
\draw[gp path] (1.504,4.147)--(11.947,4.147);
\gpcolor{color=gp lt color border}
\gpsetlinetype{gp lt border}
\gpsetdashtype{gp dt solid}
\gpsetlinewidth{1.00}
\draw[gp path] (1.504,4.147)--(1.684,4.147);
\draw[gp path] (11.947,4.147)--(11.767,4.147);
\node[gp node right] at (1.5,4.147) {\scriptsize $800$};
\gpcolor{color=gp lt color axes}
\gpsetlinetype{gp lt axes}
\gpsetdashtype{gp dt axes}
\gpsetlinewidth{0.50}
\draw[gp path] (1.504,4.928)--(11.947,4.928);
\gpcolor{color=gp lt color border}
\gpsetlinetype{gp lt border}
\gpsetdashtype{gp dt solid}
\gpsetlinewidth{1.00}
\draw[gp path] (1.504,4.928)--(1.684,4.928);
\draw[gp path] (11.947,4.928)--(11.767,4.928);
\node[gp node right] at (1.5,4.928) {\scriptsize $1000$};
\gpcolor{color=gp lt color axes}
\gpsetlinetype{gp lt axes}
\gpsetdashtype{gp dt axes}
\gpsetlinewidth{0.50}
\draw[gp path] (1.504,5.708)--(11.947,5.708);
\gpcolor{color=gp lt color border}
\gpsetlinetype{gp lt border}
\gpsetdashtype{gp dt solid}
\gpsetlinewidth{1.00}
\draw[gp path] (1.504,5.708)--(1.684,5.708);
\draw[gp path] (11.947,5.708)--(11.767,5.708);
\node[gp node right] at (1.5,5.708) {\scriptsize $1200$};
\gpcolor{color=gp lt color axes}
\gpsetlinetype{gp lt axes}
\gpsetdashtype{gp dt axes}
\gpsetlinewidth{0.50}
\draw[gp path] (1.504,6.489)--(11.947,6.489);
\gpcolor{color=gp lt color border}
\gpsetlinetype{gp lt border}
\gpsetdashtype{gp dt solid}
\gpsetlinewidth{1.00}
\draw[gp path] (1.504,6.489)--(1.684,6.489);
\draw[gp path] (11.947,6.489)--(11.767,6.489);
\node[gp node right] at (1.5,6.489) {\scriptsize $1400$};
\gpcolor{color=gp lt color axes}
\gpsetlinetype{gp lt axes}
\gpsetdashtype{gp dt axes}
\gpsetlinewidth{0.50}
\draw[gp path] (1.504,7.270)--(11.947,7.270);
\gpcolor{color=gp lt color border}
\gpsetlinetype{gp lt border}
\gpsetdashtype{gp dt solid}
\gpsetlinewidth{1.00}
\draw[gp path] (1.504,7.270)--(1.684,7.270);
\draw[gp path] (11.947,7.270)--(11.767,7.270);
\node[gp node right] at (1.5,7.270) {\scriptsize $1600$};
\gpcolor{color=gp lt color axes}
\gpsetlinetype{gp lt axes}
\gpsetdashtype{gp dt axes}
\gpsetlinewidth{0.50}
\draw[gp path] (1.504,8.051)--(11.947,8.051);
\gpcolor{color=gp lt color border}
\gpsetlinetype{gp lt border}
\gpsetdashtype{gp dt solid}
\gpsetlinewidth{1.00}
\draw[gp path] (1.504,8.051)--(1.684,8.051);
\draw[gp path] (11.947,8.051)--(11.767,8.051);
\node[gp node right] at (1.5,8.051) {\scriptsize $1800$};
\gpcolor{color=gp lt color axes}
\gpsetlinetype{gp lt axes}
\gpsetdashtype{gp dt axes}
\gpsetlinewidth{0.50}
\draw[gp path] (2.026,0.985)--(2.026,8.441);
\gpcolor{color=gp lt color border}
\gpsetlinetype{gp lt border}
\gpsetdashtype{gp dt solid}
\gpsetlinewidth{1.00}
\draw[gp path] (2.026,0.985)--(2.026,1.165);
\draw[gp path] (2.026,8.441)--(2.026,8.261);
\node[gp node center] at (2.026,0.677) {\scriptsize $10$};
\gpcolor{color=gp lt color axes}
\gpsetlinetype{gp lt axes}
\gpsetdashtype{gp dt axes}
\gpsetlinewidth{0.50}
\draw[gp path] (3.070,0.985)--(3.070,8.441);
\gpcolor{color=gp lt color border}
\gpsetlinetype{gp lt border}
\gpsetdashtype{gp dt solid}
\gpsetlinewidth{1.00}
\draw[gp path] (3.070,0.985)--(3.070,1.165);
\draw[gp path] (3.070,8.441)--(3.070,8.261);
\node[gp node center] at (3.070,0.677) {\scriptsize $20$};
\gpcolor{color=gp lt color axes}
\gpsetlinetype{gp lt axes}
\gpsetdashtype{gp dt axes}
\gpsetlinewidth{0.50}
\draw[gp path] (4.115,0.985)--(4.115,8.441);
\gpcolor{color=gp lt color border}
\gpsetlinetype{gp lt border}
\gpsetdashtype{gp dt solid}
\gpsetlinewidth{1.00}
\draw[gp path] (4.115,0.985)--(4.115,1.165);
\draw[gp path] (4.115,8.441)--(4.115,8.261);
\node[gp node center] at (4.115,0.677) {\scriptsize $30$};
\gpcolor{color=gp lt color axes}
\gpsetlinetype{gp lt axes}
\gpsetdashtype{gp dt axes}
\gpsetlinewidth{0.50}
\draw[gp path] (5.159,0.985)--(5.159,8.441);
\gpcolor{color=gp lt color border}
\gpsetlinetype{gp lt border}
\gpsetdashtype{gp dt solid}
\gpsetlinewidth{1.00}
\draw[gp path] (5.159,0.985)--(5.159,1.165);
\draw[gp path] (5.159,8.441)--(5.159,8.261);
\node[gp node center] at (5.159,0.677) {\scriptsize $40$};
\gpcolor{color=gp lt color axes}
\gpsetlinetype{gp lt axes}
\gpsetdashtype{gp dt axes}
\gpsetlinewidth{0.50}
\draw[gp path] (6.203,0.985)--(6.203,8.441);
\gpcolor{color=gp lt color border}
\gpsetlinetype{gp lt border}
\gpsetdashtype{gp dt solid}
\gpsetlinewidth{1.00}
\draw[gp path] (6.203,0.985)--(6.203,1.165);
\draw[gp path] (6.203,8.441)--(6.203,8.261);
\node[gp node center] at (6.203,0.677) {\scriptsize $50$};
\gpcolor{color=gp lt color axes}
\gpsetlinetype{gp lt axes}
\gpsetdashtype{gp dt axes}
\gpsetlinewidth{0.50}
\draw[gp path] (7.248,0.985)--(7.248,8.441);
\gpcolor{color=gp lt color border}
\gpsetlinetype{gp lt border}
\gpsetdashtype{gp dt solid}
\gpsetlinewidth{1.00}
\draw[gp path] (7.248,0.985)--(7.248,1.165);
\draw[gp path] (7.248,8.441)--(7.248,8.261);
\node[gp node center] at (7.248,0.677) {\scriptsize $60$};
\gpcolor{color=gp lt color axes}
\gpsetlinetype{gp lt axes}
\gpsetdashtype{gp dt axes}
\gpsetlinewidth{0.50}
\draw[gp path] (8.292,0.985)--(8.292,8.441);
\gpcolor{color=gp lt color border}
\gpsetlinetype{gp lt border}
\gpsetdashtype{gp dt solid}
\gpsetlinewidth{1.00}
\draw[gp path] (8.292,0.985)--(8.292,1.165);
\draw[gp path] (8.292,8.441)--(8.292,8.261);
\node[gp node center] at (8.292,0.677) {\scriptsize $70$};
\gpcolor{color=gp lt color axes}
\gpsetlinetype{gp lt axes}
\gpsetdashtype{gp dt axes}
\gpsetlinewidth{0.50}
\draw[gp path] (9.336,0.985)--(9.336,8.441);
\gpcolor{color=gp lt color border}
\gpsetlinetype{gp lt border}
\gpsetdashtype{gp dt solid}
\gpsetlinewidth{1.00}
\draw[gp path] (9.336,0.985)--(9.336,1.165);
\draw[gp path] (9.336,8.441)--(9.336,8.261);
\node[gp node center] at (9.336,0.677) {\scriptsize $80$};
\gpcolor{color=gp lt color axes}
\gpsetlinetype{gp lt axes}
\gpsetdashtype{gp dt axes}
\gpsetlinewidth{0.50}
\draw[gp path] (10.381,0.985)--(10.381,8.441);
\gpcolor{color=gp lt color border}
\gpsetlinetype{gp lt border}
\gpsetdashtype{gp dt solid}
\gpsetlinewidth{1.00}
\draw[gp path] (10.381,0.985)--(10.381,1.165);
\draw[gp path] (10.381,8.441)--(10.381,8.261);
\node[gp node center] at (10.381,0.677) {\scriptsize $90$};
\gpcolor{color=gp lt color axes}
\gpsetlinetype{gp lt axes}
\gpsetdashtype{gp dt axes}
\gpsetlinewidth{0.50}
\draw[gp path] (11.425,0.985)--(11.425,8.441);
\gpcolor{color=gp lt color border}
\gpsetlinetype{gp lt border}
\gpsetdashtype{gp dt solid}
\gpsetlinewidth{1.00}
\draw[gp path] (11.425,0.985)--(11.425,1.165);
\draw[gp path] (11.425,8.441)--(11.425,8.261);
\node[gp node center] at (11.425,0.677) {\scriptsize $100$};
\draw[gp path] (1.504,8.441)--(1.504,0.985)--(11.947,0.985)--(11.947,8.441)--cycle;

% Antes da legenda, adicione:
\fill[white, opacity=0.95] (2.8,6.25) rectangle (5.5,7.75);
\draw[black, very thin] (2.8,6.25) rectangle (5.5,7.75);

\node[gp node right] at (4,7.25) {$t_0[n]$};
\gpcolor{rgb color={1.000,0.000,0.000}}
\gpsetlinewidth{2.00}
\draw[gp path] (4.25,7.25)--(5.25,7.25);
\draw[gp path] (2.026,1.033)--(3.070,1.041)--(4.115,1.048)--(5.159,1.056)--(6.203,1.064)%
  --(7.248,1.072)--(8.292,1.080)--(9.336,1.088)--(10.381,1.095)--(11.425,1.103);
\gpsetpointsize{6.00}
\gp3point{gp mark 7}{}{(2.026,1.033)}
\gp3point{gp mark 7}{}{(3.070,1.041)}
\gp3point{gp mark 7}{}{(4.115,1.048)}
\gp3point{gp mark 7}{}{(5.159,1.056)}
\gp3point{gp mark 7}{}{(6.203,1.064)}
\gp3point{gp mark 7}{}{(7.248,1.072)}
\gp3point{gp mark 7}{}{(8.292,1.080)}
\gp3point{gp mark 7}{}{(9.336,1.088)}
\gp3point{gp mark 7}{}{(10.381,1.095)}
\gp3point{gp mark 7}{}{(11.425,1.103)}
\gp3point{gp mark 7}{}{(4.75,7.25)}
\gpcolor{color=gp lt color border}
\node[gp node right] at (4,6.75) {$t_1[n]$};
\gpcolor{rgb color={0.000,0.000,1.000}}
\draw[gp path] (4.25,6.75)--(5.25,6.75);
\draw[gp path] (2.026,1.093)--(3.070,1.300)--(4.115,1.648)--(5.159,2.137)--(6.203,2.766)%
  --(7.248,3.536)--(8.292,4.447)--(9.336,5.497)--(10.381,6.689)--(11.425,8.021);
\gp3point{gp mark 7}{}{(2.026,1.093)}
\gp3point{gp mark 7}{}{(3.070,1.300)}
\gp3point{gp mark 7}{}{(4.115,1.648)}
\gp3point{gp mark 7}{}{(5.159,2.137)}
\gp3point{gp mark 7}{}{(6.203,2.766)}
\gp3point{gp mark 7}{}{(7.248,3.536)}
\gp3point{gp mark 7}{}{(8.292,4.447)}
\gp3point{gp mark 7}{}{(9.336,5.497)}
\gp3point{gp mark 7}{}{(10.381,6.689)}
\gp3point{gp mark 7}{}{(11.425,8.021)}
\gp3point{gp mark 7}{}{(4.75,6.75)}
\gpcolor{color=gp lt color border}
\gpsetlinewidth{1.00}
\draw[gp path] (1.504,8.441)--(1.504,0.985)--(11.947,0.985)--(11.947,8.441)--cycle;
\node[gp node center,rotate=-270.0] at (0.292,4.713) {$t[n]$};
\node[gp node center] at (6.725,0.215) {$n$};
%% coordinates of the plot area
\gpdefrectangularnode{gp plot 1}{\pgfpoint{1.504cm}{0.985cm}}{\pgfpoint{11.947cm}{8.441cm}}
\end{tikzpicture}
%% gnuplot variables
}
    \vspace{-40pt}
    \caption{Comparação entre $t_1[n]$ e $t_0[n]$}
    \vspace{-10pt}
    \label{fig:placeholder}
\end{figure}

Perceba que o gráfico de $t_1[n]$ se distancia tanto do de $t_0[n]$ conforme $n$ cresce que o segundo chega a parecer constante em $t = 0$. Esse resultado é coerente, pois $t_1[n]$ é uma função de segundo grau e $t_0[n]$, de primeiro. De fato, no limite, temos
\[
\lim_{n\to\infty} \frac{t_0[n]}{t_1[n]} =\lim_{n\to\infty} \frac{0,2n + 0,26}{0,18 n^2 - 0,08n + 0,36} = 0
\]

\subsection{Segundo Experimento}
O segundo experimento tem como objetivo compreender a compilação cruzada e a geração de código Assembly a partir de C, com foco em RISC-V. Utilizamos o site Compiler Explorer para compilar os códigos em C com diferentes diretivas.

\subsubsection{Tarefa I}
% Enunciado da primeira tarefa da 2
\begin{tcolorbox}[title=Enunciado, colback=blue!5!white, colframe=blue!75!black]
Dado o programa sortc.c, compile-o com a diretiva -O0 e obtenha o arquivo sortc.s. Indique as modificações necessárias no código Assembly gerado para que possa ser executado corretamente no RARS.
\end{tcolorbox}

A primeira vista, percebe-se que o compilador não utiliza as diretivas \cod{.data} e \cod{.text}, que são necessárias para a montagem no RARS. Além disso, no simulador, as intruções começam a ser executadas a partir do topo da região \cod{.text}, dessa forma, seria necessário deslocar o procedimento \cod{main} ou chamá-lo no início dessa seção, além de incluir uma \textit{syscall} com \cod{a7=10} para terminar o programa. Ademais, sem incluir a biblioteca padrão do C, funções como \cod{printf} e \cod{putchar} não podem ser utilizadas; seguindo a dica fornecida no roteiro, utilizamos a função \cod{show}, definida no programa \cod{sort.s}.

Com as correções acima, o programa já pode ser montado pelo RARS, porém, ainda há um erro em tempo de execução: \cod{Error in : Instruction load access error}. O erro provavelmente se deve à má utilização do par \cod{lui} + \cod{addi} para acessar a memória, pois sua substituição por \cod{la} resolveu o problema, tornando o programa totalmente funcional.

\subsubsection{Tarefa II}
% Enunciado da segunda tarefa da 2
\begin{tcolorbox}[title=Enunciado, colback=blue!5!white, colframe=blue!75!black]
Compile o programa \cod{sortc\_mod.c} e, com a ajuda do RARS, monte uma tabela comparativa com o número total de instruções executadas pelo programa todo, e o tamanho em bytes dos códigos em linguagem de máquina gerados para cada diretiva de otimização da compilação \{\cod{-O0}, \cod{-O3}, \cod{-Os}\}. Compare ainda com os resultados obtidos no item 1.1) com o programa \cod{sort.s}, que foi implementado diretamente em Assembly. Analise os resultados obtidos usando o mesmo vetor de entrada.  
\end{tcolorbox}

% Tabela com os resultados das otimizações
\begin{table}[H]
\centering
\label{tab:results}
\small
\setlength{\tabcolsep}{3pt}
\begin{tabular}{@{}ccc@{}}
    \toprule
    \textbf{Diretriz} & \textbf{Instruções executadas} & \textbf{Tamanho (bytes)}\\
    \midrule
    -O0 & 10.081 & 496\\
    -Os & 3.905 & 312\\
    -O3 & 2.181 & 268\\
    \cod{sort.s} & 3.746 & 296\\
    \bottomrule
\end{tabular}
\caption{Instruções executadas para cada diretiva de compilação}
\vspace{-10pt}
\end{table}

% Comentários sobre a tabela 
Os dados de desempenho para as diferentes diretrizes de compilação fornecem observações interessantes sobre as estratégias de otimização do GCC. Ao comparar os resultados é possível observar relações entre eficiência e tamanho do código. 

A diretiva -O0, que compila sem otimização, é a que gera um código com mais instruções executadas (10.081) e o maior tamanho em bytes (496).

A \textit{flag} -Os, que prioriza a redução do tamanho do código, reduziu o número de instruções (3.905) e o tamanho do código (312 bytes) significativamente em relação à -O0, mas não conseguiu superar em nenhum dos quesitos as otimizações de -O3 ou o código escrito direto em Assembly \cod{sort.s}. 

A \textit{flag} -O3, que representa o nível máximo de otimização, mostrou resultados mais eficientes, executando menos instruções que todas as outras opções (2181) e com o menor tamanho (268 bytes). Esse desempenho é alcançado com técnicas como eliminação de código redundante e reordenação de instruções, e evidencia a eficácia da otimização do compilador em gerar código enxuto.

O código \cod{sort.s}, escrito manualmente em Assembly, apresentou 3746 instruções e um tamanho de 296 bytes, ficando entre -Os e -O3. Este resultado demonstra que programar em Assembly pode ser eficiente, mas, para bater de frente com compiladores modernos e otimizados, é necessário conhecimento e grande esforço.

\subsubsection{Tarefa III}
% Enunciado da terceira tarefa da 2
\begin{tcolorbox}[title=Enunciado, colback=blue!5!white, colframe=blue!75!black]
Pesquise na internet e explique as diferenças entre as otimizações \cod{-O0}, \cod{-O1}, \cod{-O2}, \cod{-O3} e \cod{-Os}.  
\end{tcolorbox}

% Otimizações
\begin{itemize}
    \item -O0 (Sem otimização): Configuração padrão do compilador, quando nenhuma otimização é especificada. Ao compilar com -O0, o código gerado é praticamente uma tradução direta do código fonte para assembly, o que facilita a depuração. A estrutura do código original é preservada, mantendo todas as variáveis e evitando rearranjos ou remoções de código, para que possa ser feito um mapeamento um-para-um;
    \item -O1: Otimização com o objetivo de reduzir tamanho do código e tempo de execução sem aumentar o tempo de compilação. Entre as otimizações estão a remoção de código que nunca são executados (código morto), e simplificação de expressões;
    \item -O2: Otimiza mais ainda, incluindo todas as otimizações de -O1 e ativando outras como inline de funções pequenas para eliminar overhead de chamadas e análise de fluxo de dados. Aumenta tanto o tempo de compilação quanto o desempenho do código gerado;
    \item -O3: Todas as otimizações de -O2 são ativadas, além de outras transformações que maximizam o desempenho. Esta diretriz aplica técnicas como inlining de funções, unrolling de loops, onde o compilador replica sistematicamente o corpo dos loops para reduzir significativamente o overhead de controle e vetorização. Esta abordagem agressiva apresenta entretanto um \textit{trade-off} significativo, o código gerado pode ter um tempo de compilação maior, e um aumento em tamanho que acaba impactando negativamente a utilização da memória cache;
    \item -Os: Otimização projetada para sistemas embarcados ou com recursos limitados de memória. -Os aplica otimizações de -O2 que não aumentam o tamanho do código e desativa as que tendem a expandir o código. São utilizadas técnicas como eliminação de código não utilizado, seleção de instruções mais curtas e reorganização de código, visando gerar um executável menor.
\end{itemize}

\newpage

\subsection{Terceiro Experimento}
O terceiro experimento consiste em implementar, em Assembly RISC-V, a Transformada Discreta de Fourier (DFT), que realiza uma mudança de base entre duas bases ortonormais de dimensão $N$: a base temporal (amostras) composta por $N$ impulsos unitários deslocados $\delta[n-m],\ m \in \{0, \dots, N-1\}$ para a base frequencial composta por $N$ exponenciais complexas $e^{-i\frac{2\pi}{N}kn},\ k \in \{0, \dots, N-1\}$. A DFT é dada pela seguinte fórmula:
\vspace{-10pt}
\begin{equation}
    X[k] = \sum_{n=0}^{N-1} x[n]e^{-i\frac{2\pi}{N}kn}
    \vspace{-8pt}
\end{equation}
em que $x[n]$ é o $n$-ésimo termo do sinal (sequência) $x$ no domínio do tempo, $X[k]$ é $k$-ésimo termo desse sinal no domínio da frequência e $N$ é a largura do sinal.

\subsubsection{Tarefa I}

\begin{tcolorbox}[title=Enunciado, colback=blue!5!white, colframe=blue!75!black]
Escreva um procedimento que receba um ângulo em radianos $\theta$ (em fa0) e retorne $\cos{\theta}$ (em fa0) e $\sin{\theta}$ (em fa1).
\end{tcolorbox}

Diferentemente das linguagens de alto nível, não existem funções trigonométricas nativas em Assembly. Consequentemente, é necessário utilizar séries de Taylor, séries polinomiais capazes de aproximar funções mais complicadas. No limite, com um somatório de infinitos termos, essas séries se igualam à função original, pelo menos em algum intervalo. As séries de Maclaurin para o cosseno e o seno são as seguintes:
\begin{equation*}
    \cos \theta = \sum_{n=0}^\infty \frac{(-1)^n}{(2n)!} \theta^{2n}  \quad \quad \sin \theta = \sum_{n=0}^\infty \frac{(-1)^n}{(2n+1)!} \theta^{2n+1}
\end{equation*}

Ingenuamente, poderíamos pensar em implementar o procedimento pedido por meio de dois loops separados, uma função que calcula o fatorial e outra que calcula exponenciais. Essa abordagem, porém, é excessivamente cara. A fim de otimizar o algoritmo é interessante perceber a seguinte relação:
\begin{align*}
    e^{i\theta} &= \sum_{n=0}^\infty \frac{(i\theta)^n}{n!} = \frac{(i\theta)^0}{0!} + \frac{(i\theta)^1}{1!} + \frac{(i\theta)^2}{2!} + \frac{(i\theta)^3}{3!} + \frac{(i\theta)^4}{4!} + \frac{(i\theta)^5}{5!} + \dots\\
    &= 1 + i\theta - \frac{\theta^2}{2!} - i \frac{\theta^3}{3!} + \frac{\theta^4}{4!} + i\frac{\theta^5}{5!} - \dots \\
    &= \left(1 - \frac{\theta^2}{2!} + \frac{\theta^4}{4!} - \dots \right)
    + i \left(\theta - \frac{\theta^3}{3!} + \frac{\theta^5}{5!} - \dots \right) \\
    &= \left(\sum_{n=0}^\infty \frac{(-1)^n}{(2n)!} \theta^{2n}\right)
    + i \left(\sum_{n=0}^\infty \frac{(-1)^n}{(2n+1)!} \theta^{2n+1}\right) \\
    &= \cos{\theta} + i\sin{\theta}
\end{align*}
Ou seja, além de termos demonstrado a igualdade de Euler, percebemos que se pode utilizar um único loop para calcular simultaneamente seno e cosseno (o loop referente à série da exponencial complexa), basta definir, pela paridade do índice do somatório, em qual registrador vamos acumular o resultado em cada iteração (\cod{fa0} para índices pares e \cod{fa1} para índices ímpares).

Teoricamente, a implementação do algoritmo \cod{SINCOS} conforme descrito acima (pela série de Taylor) seria suficiente. No entanto, na prática, apesar da série de Taylor convergir para qualquer ângulo, a aproximação por um número finito de termos é imprecisa para ângulos grandes e não basta aumentar o número de iterações (de termos da série) pois o cálculo de potências elevadas pode causar \textit{overflow}. Dessa forma, a fim de melhorar as aproximações, modulamos o ângulo de forma que ele fique sempre no intervalo $\left[-\frac{\pi}{2}, \frac{\pi}{2}\right]$. De acordo com a figura a seguir, isso preserva o seno, mas altera o sinal do cosseno; precisamos, portanto, criar uma \textit{flag} para inverter o cosseno.

\begin{figure}[H]
    \centering
    \definecolor{azul}{RGB}{0,0,191}
\begin{minipage}{1\textwidth}
\begin{tikzpicture}[scale=3]
    \usetikzlibrary{patterns}

    % Preenchimento do lado direito do círculo
    \fill[pattern=north west lines, pattern color=lightgray] (0,0) -- (90:1) arc (90:-90:1) -- cycle;

    % Eixos coordenados
    \draw[->, thick] (-1.2,0) -- (1.2,0) node[below] {$\Re$};
    \draw[->, thick] (0,-1.2) -- (0,1.2) node[left] {$\Im$};
    
    % Círculo unitário
    \draw[thick] (0,0) circle (1);
    
    % Ângulo genérico
    \def\ang{45} % Pode alterar este valor
    \coordinate (O) at (0,0);
    \coordinate (M) at (0,-1);
    \coordinate (B) at (0, 1);
    \coordinate (P) at ({\ang}:1);
    \coordinate (P2) at ({180-\ang}:1);
    
    % Projeções
    \draw[dashed, darkgray] (P) -- ({cos(\ang)},0) node[black,below] {\footnotesize$-\cos \theta$};
    \draw[dashed, darkgray] (P) -- (0,{sin(\ang)}) node[black,above left] {\footnotesize$\sin \theta$};
    \draw[dashed, darkgray] (P2) -- ({-cos(\ang)},0) node[black,below] {\footnotesize$\cos \theta$};
    \draw[dashed, darkgray] (P2) -- (0,{sin(\ang)});
    
    % Raio
    \draw[->, dashed, azul] (O) -- (P);
    \draw[->, thick, azul] (O) -- (P2);
    
    % Ângulo
    \draw (0.1,0) arc (0:{180-\ang}:0.1);
    \node at ({0.5*(180-\ang)}:0.15) {\scriptsize$\theta$};
    \draw[dashed] (0.15,0) arc (0:{\ang}:0.15);
    \node at ({0.25*\ang}:0.3) {\scriptsize$\pi - \theta$};

    \node[below right] at (M) {\footnotesize$-\frac{\pi}{2}$};
    \node[above right] at (B) {\footnotesize$\frac{\pi}{2}$};

\end{tikzpicture}
\begin{tikzpicture}[scale=3]
    \usetikzlibrary{patterns}

    % Preenchimento do lado direito do círculo
    \fill[pattern=north west lines, pattern color=lightgray] (0,0) -- (90:1) arc (90:-90:1) -- cycle;

    % Eixos coordenados
    \draw[->, thick] (-1.2,0) -- (1.2,0) node[below] {$\Re$};
    \draw[->, thick] (0,-1.2) -- (0,1.2) node[left] {$\Im$};
    
    % Círculo unitário
    \draw[thick] (0,0) circle (1);
    
    % Ângulo genérico
    \def\ang{45} % Pode alterar este valor
    \coordinate (O) at (0,0);
    \coordinate (M) at (0,-1);
    \coordinate (B) at (0, 1);
    \coordinate (P) at ({-\ang}:1);
    \coordinate (P2) at ({180+\ang}:1);
    
    % Projeções
    \draw[dashed, darkgray] (P) -- ({cos(\ang)},0) node[black,above] {\footnotesize$-\cos \theta$};
    \draw[dashed, darkgray] (P) -- (0,{-sin(\ang)}) node[black,below left] {\footnotesize$\sin \theta$};
    \draw[dashed, darkgray] (P2) -- ({-cos(\ang)},0) node[black,above] {\footnotesize$\cos \theta$};
    \draw[dashed, darkgray] (P2) -- (0,{-sin(\ang)});
    
    % Raio
    \draw[->, dashed, azul] (O) -- (P);
    \draw[->, thick, azul] (O) -- (P2);
    
    % Ângulo
    \draw (0.1,0) arc (0:{180+\ang}:0.1);
    \node at ({0.5*(180+\ang)}:0.15) {\scriptsize$\theta$};
    \draw[dashed] (0.15,0) arc (0:{-\ang}:0.15);
    \node at ({0.3*(-\ang)}:0.3) {\scriptsize$\pi - \theta$};

    \node[below right] at (M) {\footnotesize$-\frac{\pi}{2}$};
    \node[above right] at (B) {\footnotesize$\frac{\pi}{2}$};

\end{tikzpicture}
\end{minipage}
\begin{minipage}{1\textwidth}
\begin{tikzpicture}[scale=3]
    \usetikzlibrary{patterns}

    % Preenchimento do lado direito do círculo
    \fill[pattern=north west lines, pattern color=lightgray] (0,0) -- (90:1) arc (90:-90:1) -- cycle;

    % Eixos coordenados
    \draw[->, thick] (-1.2,0) -- (1.2,0) node[below] {$\Re$};
    \draw[->, thick] (0,-1.2) -- (0,1.2) node[left] {$\Im$};
    
    % Círculo unitário
    \draw[thick] (0,0) circle (1);
    
    % Ângulo genérico
    \def\ang{-45} % Pode alterar este valor
    \coordinate (O) at (0,0);
    \coordinate (M) at (0,-1);
    \coordinate (B) at (0, 1);
    \coordinate (P) at ({\ang}:1);
    \coordinate (P2) at ({-180-\ang}:1);
    
    % Projeções
    \draw[dashed, darkgray] (P) -- ({cos(\ang)},0) node[black,above] {\footnotesize$-\cos \theta$};
    \draw[dashed, darkgray] (P) -- (0,{sin(\ang)}) node[black,above left] {\footnotesize$\sin \theta$};
    \draw[dashed, darkgray] (P2) -- ({-cos(\ang)},0) node[black,above] {\footnotesize$\cos \theta$};
    \draw[dashed, darkgray] (P2) -- (0,{sin(\ang)});
    
    % Raio
    \draw[->, dashed, azul] (O) -- (P);
    \draw[->, thick, azul] (O) -- (P2);
    
    % Ângulo
    \draw (0.1,0) arc (0:{-180-\ang}:0.1);
    \node at ({0.5*(-180-\ang)}:0.15) {\scriptsize$\theta$};
    \draw[dashed] (0.15,0) arc (0:{\ang}:0.15);
    \node at ({0.25*\ang}:0.32) {\scriptsize$-\pi - \theta$};

    \node[below right] at (M) {\footnotesize$-\frac{\pi}{2}$};
    \node[above right] at (B) {\footnotesize$\frac{\pi}{2}$};

\end{tikzpicture}
\begin{tikzpicture}[scale=3]
    \usetikzlibrary{patterns}

    % Preenchimento do lado direito do círculo
    \fill[pattern=north west lines, pattern color=lightgray] (0,0) -- (90:1) arc (90:-90:1) -- cycle;

    % Eixos coordenados
    \draw[->, thick] (-1.2,0) -- (1.2,0) node[below] {$\Re$};
    \draw[->, thick] (0,-1.2) -- (0,1.2) node[left] {$\Im$};
    
    % Círculo unitário
    \draw[thick] (0,0) circle (1);
    
    % Ângulo genérico
    \def\ang{-225} % Pode alterar este valor
    \coordinate (O) at (0,0);
    \coordinate (M) at (0,-1);
    \coordinate (B) at (0, 1);
    \coordinate (P) at ({\ang}:1);
    \coordinate (P2) at ({-180-\ang}:1);
    
    % Projeções
    \draw[dashed, darkgray] (P) -- ({cos(\ang)},0) node[black,below] {\footnotesize$\cos \theta$};
    \draw[dashed, darkgray] (P) -- (0,{sin(\ang)}) node[black,above left] {\footnotesize$\sin \theta$};
    \draw[dashed, darkgray] (P2) -- ({-cos(\ang)},0) node[black,below] {\footnotesize$-\cos \theta$};
    \draw[dashed, darkgray] (P2) -- (0,{sin(\ang)});
    
    % Raio
    \draw[->, thick, azul] (O) -- (P);
    \draw[->, dashed, azul] (O) -- (P2);
    
    % Ângulo
    \draw (0.1,0) arc (0:{\ang}:0.1);
    \node at ({0.5*(\ang)}:0.15) {\scriptsize$\theta$};
    \draw[dashed] (0.15,0) arc (0:{-180-\ang}:0.15);
    \node at ({0.3*(-180-\ang)}:0.32) {\scriptsize$-\pi - \theta$};

    \node[below right] at (M) {\footnotesize$-\frac{\pi}{2}$};
    \node[above right] at (B) {\footnotesize$\frac{\pi}{2}$};

\end{tikzpicture}
\end{minipage}
    \caption{Representação gráfica de quatro casos de interesse da normalização realizada}
\end{figure}

\subsubsection{Tarefa II}
\begin{tcolorbox}[title=Enunciado, colback=blue!5!white, colframe=blue!75!black]
Escreva um procedimento em Assembly RISC-V com a seguinte definição 
\begin{center}
\vspace{-10pt}
\begin{minipage}{0.8\textwidth}
\begin{minted}[style=tango]{c}
void DFT(float *x, float *X_real, float *X_imag, int N)
\end{minted}
\end{minipage}
\vspace{-10pt}
\end{center}
que, dado o endereço do vetor $x[n]$ de floats (em \cod{a0}) de tamanho $N$ na memória, os endereços dos espaços reservados para o vetor complexo $X[k]$ (parte real e parte imaginária) (em \cod{a1} e \cod{a2}) e o número de pontos $N$ (em \cod{a3}), calcule a DFT de $N$ pontos de $x[n]$ e coloque o resultado no espaço alocado para $X_{real}[k]$ e $X_{imag}[k]$.
\end{tcolorbox}

Tendo em mãos o procedimento da última tarefa, a implementação da DFT torna-se conceitualmente simples. O procedimento consiste em criar um loop que invoca a função \cod{SINCOS} - que retorna as componentes real e imaginária de $e^{-i\frac{2\pi}{N}kn}$ nos registradores \cod{fa0} e \cod{fa1}, respectivamente - multiplicar esses valores por $x[n]$ e armazenar os resultados nas posições de memória correspondentes. A complexidade reside apenas no gerenciamento operacional dos registradores, pois, como temos procedimentos aninhados, é necessário utilizar a \textit{stack}.

\subsubsection{Tarefa III}
\begin{tcolorbox}[title=Enunciado, colback=blue!5!white, colframe=blue!75!black, breakable=true]
Escreva um programa \cod{main} que defina no \cod{.data} o vetor $x[n]$, o espaço para o vetor $X[K]$, o valor de $N$ e chame o procedimento DFT.
\begin{code}
    \begin{minted}[bgcolor=blue!5!white,linenos=false]{asm}
.data 
N:      .word   8  
x:      .float  1.0, 1.0, 1.0, 1.0, 1.0, 1.0, 1.0, 1.0 
X_real: .float  0.0, 0.0, 0.0, 0.0, 0.0, 0.0, 0.0, 0.0 
X_imag: .float  0.0, 0.0, 0.0, 0.0, 0.0, 0.0, 0.0, 0.0 
.text 
...
jal DFT 
... 
    \end{minted}
\end{code}
A seguir, apresente no console a saída dos N pontos no formato: 
\begin{minted}[fontsize=\scriptsize]{bash}
x[n]    X[k] 
1.0     8.0+0.0i      
1.0     0.0+0.0i 
1.0     0.0+0.0i 
1.0     0.0+0.0i 
1.0     0.0+0.0i 
1.0     0.0+0.0i 
1.0     0.0+0.0i 
1.0     0.0+0.0i 
\end{minted}
\end{tcolorbox}

Essa \cod{main} consiste em uma chamada à função DFT e um loop para exibir \textit{floats} e caracteres no console (além de uma pequena lógica condicional para não exibir o `+' quando a parte imaginária é negativa). O resultado, exibido a seguir, não ficou igual ao do exemplo do enunciado, em que todos os termos de $X[k]$, com excessão do primeiro, são $0.0$, mas acreditamos que isso se deva a dois fatores: à imperfeição da aproximação por polinômios de Taylor e à precisão de máquina. De fato, como os maiores resultados foram da ordem de $10^{-6}$, essas são justificativas razoáveis.

\begin{tcolorbox}[colback=blue!5!white, colframe=blue!75!black, boxrule=0.5pt, arc=2pt, left=4pt, right=4pt, top=4pt, bottom=4pt, width=0.4\textwidth, center]
\begin{minted}[fontsize=\scriptsize]{bash}
x[n]	X[k]
1.0	8.0+0.0i
1.0	-7.748604E-7+0.0i
1.0	-6.727769E-7+2.3841858E-7i
1.0	3.5762787E-7-4.7683716E-7i
1.0	0.0-7.1525574E-7i
1.0	-2.7418137E-6-1.1324883E-6i
1.0	-1.6589261E-6+1.9073486E-6i
1.0	-2.0861626E-6-1.1920929E-7i
\end{minted}
\end{tcolorbox}

\subsubsection{Tarefa IV}
\begin{tcolorbox}[title=Enunciado, colback=blue!5!white, colframe=blue!75!black]
Calcule a DFT dos seguintes vetores $x[n]$, com $N=8$
\begin{minted}[bgcolor=blue!5!white,linenos=false]{asm}
x1: .float  1.0, 0.0, 0.0, 0.0, 0.0, 0.0, 0.0, 0.0 
x2: .float  1.0, 0.7071, 0.0, -0.7071, -1.0, -0.7071, 0.0, 0.7071 
x3: .float  0.0, 0.7071. 1.0,  0.7071, 0.0, -0.7071, -1.0, -0.7071 
x4: .float  1.0, 1.0, 1.0, 1.0, 0.0, 0.0, 0.0, 0.0 
\end{minted}
\end{tcolorbox}
Utilizando o código desenvolvido nas tarefas anteriores, calculamos a DFT de cada um dos sinais. Os resultados estão exibidos a seguir, de acordo com a ordem do enunciado.

\begin{multicols}{2}
    \begin{codebox}{$x_1$}
    \begin{minted}[fontsize=\scriptsize]{bash}
x[n]           X[k]
1.0	    1.0+0.0i
0.0	    1.0+0.0i
0.0	    1.0+0.0i
0.0	    1.0+0.0i
0.0	    1.0+0.0i
0.0	    1.0+0.0i
0.0	    1.0+0.0i
0.0	    1.0+0.0i
    \end{minted}
    \end{codebox}
    
    \begin{codebox}{$x_2$}
    \begin{minted}[fontsize=\scriptsize]{bash}
x[n]            X[k]
1.0             -5.9604645E-8+0.0i
0.7071          3.999981+3.2782555E-7i
0.0             9.760695E-7+0.0i
-0.7071	 1.963973E-5-7.4505806E-7i
-1.0            0.0-5.057573E-7i
-0.7071         1.8894672E-5+2.9802322E-8i
0.0	     -8.34465E-7+0.0i
0.7071	  3.9999797+2.592802E-6i
    \end{minted}
    \end{codebox}
\columnbreak
    \begin{codebox}{$x_3$}
    \begin{minted}[fontsize=\scriptsize]{bash}
x[n]	    X[k]
0.0	     -5.9604645E-8+0.0i
0.7071	  5.9604645E-8+3.9999797i
1.0	     -6.668487E-7-7.1525574E-7i
0.7071	  -1.4305115E-6-1.8388033E-5i
0.0	     0.0+8.429289E-7i
-0.7071	 9.23872E-7+1.9133091E-5i
-1.0	    1.4437442E-6-9.536743E-7i
-0.7071	 3.784895E-6-3.999981i
    \end{minted}
    \end{codebox}
    
    \begin{codebox}{$x_4$}
    \begin{minted}[fontsize=\scriptsize]{bash}
x[n]	    X[k]
1.0	    4.0+0.0i
1.0	    0.9999994+2.4142132i
1.0	    -1.02313E-6-2.3841858E-7i
1.0	    0.9999994+0.4142136i
0.0	    0.0+2.3841858E-7i
0.0	    0.999999-0.4142139i
0.0	    9.851078E-9+4.7683716E-7i
0.0	    1.0000013-2.4142137i
    \end{minted}
    \end{codebox}
\end{multicols}

Novamente, percebe-se um pequeno erro nos resultados: alguns números são da ordem de $10^{-7}$, outros, extremamente próximos de 1 ou 4, que eram os valores esperados. Justificamos essas imprecisões da mesma forma que as da questão anterior: erros na aproximação por polinômios e na precisão de máquina.

\subsubsection{Tarefa V}
\begin{tcolorbox}[title=Enunciado, colback=blue!5!white, colframe=blue!75!black]
Para os sinais $x[n]$ abaixo
\begin{enumerate}[label=\alph*), itemsep=0pt]
    \item $N=8$, $x[n]=\{1.0,\ 1.0,\ 1.0,\ 0.0,\ 0.0,\ 0.0,\ 0.0 ,\ 0.0\} $
    \item $N=12$,\ $x[n]=\{1.0,\ 1.0,\ 1.0,\ 0.0,\ 0.0,\ 0.0,\ 0.0 ,\ \dots,\ 0.0\}$ 
    \item $N=16$,\ $x[n]=\{1.0,\ 1.0,\ 1.0,\ 0.0,\ 0.0,\ 0.0,\ 0.0 ,\ \dots,\ 0.0\}$ 
    \item $N=20$,\ $x[n]=\{1.0,\ 1.0,\ 1.0,\ 0.0,\ 0.0,\ 0.0,\ 0.0 ,\ \dots,\ 0.0\}$ 
    \item $N=24$,\ $x[n]=\{1.0,\ 1.0,\ 1.0,\ 0.0,\ 0.0,\ 0.0,\ 0.0 ,\ \dots,\ 0.0\}$ 
    \item $N=28$,\ $x[n]=\{1.0,\ 1.0,\ 1.0,\ 0.0,\ 0.0,\ 0.0,\ 0.0 ,\ \dots,\ 0.0\}$ 
    \item $N=32$,\ $x[n]=\{1.0,\ 1.0,\ 1.0,\ 0.0,\ 0.0,\ 0.0,\ 0.0 ,\ \dots,\ 0.0\}$ 
    \item $N=36$,\ $x[n]=\{1.0,\ 1.0,\ 1.0,\ 0.0,\ 0.0,\ 0.0,\ 0.0 ,\ \dots,\ 0.0\}$ 
    \item $N=40$,\ $x[n]=\{1.0,\ 1.0,\ 1.0,\ 0.0,\ 0.0,\ 0.0,\ 0.0 ,\ \dots,\ 0.0\}$ 
    \item $N=44$,\ $x[n]=\{1.0,\ 1.0,\ 1.0,\ 0.0,\ 0.0,\ 0.0,\ 0.0 ,\ \dots,\ 0.0\}$
\end{enumerate}
onde $\dots$ são todos valores 0.0. Para cada item, meça o tempo de execução do procedimento DFT e calcule a frequência do processador RISC-V Uniciclo simulado pelo RARS. Finalmente, faça um gráfico $t_{exec} \times N$ em escala. Quais conclusões podemos tirar dessa análise?
\end{tcolorbox}
Utilizando os registradores de status e controle do RARS, medimos o tempo de execução ($t_{exec}$) e a quantidade de instruções ($I$) do programa para cada caso. A frequência foi calculada pela fórmula
\[
f_{clock} = \frac{I \cdot CPI}{t_{exec}} \overset{CPI=1}{=} \frac{I}{t_{exec}}
\]
A seguir, os resultados:
\begin{enumerate}[label=\alph*), itemsep=0pt]
    \item $N=8 \to t_{exec} = 48ms$, $I = 15.606$, $f = 325kHz$
    \item $N=12 \to t_{exec} = 84ms$, $I = 34.999$, $f = 416kHz$
    \item $N=16 \to t_{exec} = 144ms$, $I = 62.137$, $f = 431kHz$
    \item $N=20 \to t_{exec} = 235ms$, $I = 96.981$, $f = 412kHz$
    \item $N=24 \to t_{exec} = 343ms$, $I = 139.615$, $f = 407kHz$
    \item $N=28 \to t_{exec} = 483ms$, $I = 189.955$, $f = 393kHz$
    \item $N=32 \to t_{exec} = 595ms$, $I = 248.028$, $f = 416kHz$
    \item $N=36 \to t_{exec} = 769ms$, $I = 313.777$, $f = 408kHz$
    \item $N=40 \to t_{exec} = 949ms$, $I = 387.373$, $f = 408kHz$
    \item $N=44 \to t_{exec} = 1153ms$, $I = 468.639$, $f = 406kHz$
\end{enumerate}

Com esses dados, podemos fazer o gráfico solicitado. Novamente, utilizamos o gnuplot e plotamos o seguinte gráfico:

\begin{figure}[H]
    \centering
    \begin{tikzpicture}[gnuplot]
%% generated with GNUPLOT 6.0p0 (Lua 5.4; terminal rev. Jun 2020, script rev. 118)
%% sáb 18 out 2025 14:57:35
\path (0.000,0.000) rectangle (11.500,8.750);
\gpcolor{color=gp lt color axes}
\gpsetlinetype{gp lt axes}
\gpsetdashtype{gp dt axes}
\gpsetlinewidth{0.50}
\draw[gp path] (1.504,1.162)--(11,1.162);
\gpcolor{color=gp lt color border}
\gpsetlinetype{gp lt border}
\gpsetdashtype{gp dt solid}
\gpsetlinewidth{1.00}
\draw[gp path] (1.504,1.162)--(1.684,1.162);
\draw[gp path] (11,1.162)--(10.82,1.162);
\node[gp node right] at (1.5,1.162) {\scriptsize$48$};
\gpcolor{color=gp lt color axes}
\gpsetlinetype{gp lt axes}
\gpsetdashtype{gp dt axes}
\gpsetlinewidth{0.50}
\draw[gp path] (1.504,1.389)--(11,1.389);
\gpcolor{color=gp lt color border}
\gpsetlinetype{gp lt border}
\gpsetdashtype{gp dt solid}
\gpsetlinewidth{1.00}
\draw[gp path] (1.504,1.389)--(1.684,1.389);
\draw[gp path] (11,1.389)--(10.82,1.389);
\node[gp node right] at (1.5,1.389) {\scriptsize$84$};
\gpcolor{color=gp lt color axes}
\gpsetlinetype{gp lt axes}
\gpsetdashtype{gp dt axes}
\gpsetlinewidth{0.50}
\draw[gp path] (1.504,1.769)--(11,1.769);
\gpcolor{color=gp lt color border}
\gpsetlinetype{gp lt border}
\gpsetdashtype{gp dt solid}
\gpsetlinewidth{1.00}
\draw[gp path] (1.504,1.769)--(1.684,1.769);
\draw[gp path] (11,1.769)--(10.82,1.769);
\node[gp node right] at (1.5,1.769) {\scriptsize$144$};
\gpcolor{color=gp lt color axes}
\gpsetlinetype{gp lt axes}
\gpsetdashtype{gp dt axes}
\gpsetlinewidth{0.50}
\draw[gp path] (1.504,2.344)--(11,2.344);
\gpcolor{color=gp lt color border}
\gpsetlinetype{gp lt border}
\gpsetdashtype{gp dt solid}
\gpsetlinewidth{1.00}
\draw[gp path] (1.504,2.344)--(1.684,2.344);
\draw[gp path] (11,2.344)--(10.82,2.344);
\node[gp node right] at (1.5,2.344) {\scriptsize$235$};
\gpcolor{color=gp lt color axes}
\gpsetlinetype{gp lt axes}
\gpsetdashtype{gp dt axes}
\gpsetlinewidth{0.50}
\draw[gp path] (1.504,3.026)--(11,3.026);
\gpcolor{color=gp lt color border}
\gpsetlinetype{gp lt border}
\gpsetdashtype{gp dt solid}
\gpsetlinewidth{1.00}
\draw[gp path] (1.504,3.026)--(1.684,3.026);
\draw[gp path] (11,3.026)--(10.82,3.026);
\node[gp node right] at (1.5,3.026) {\scriptsize$343$};
\gpcolor{color=gp lt color axes}
\gpsetlinetype{gp lt axes}
\gpsetdashtype{gp dt axes}
\gpsetlinewidth{0.50}
\draw[gp path] (1.504,3.911)--(11,3.911);
\gpcolor{color=gp lt color border}
\gpsetlinetype{gp lt border}
\gpsetdashtype{gp dt solid}
\gpsetlinewidth{1.00}
\draw[gp path] (1.504,3.911)--(1.684,3.911);
\draw[gp path] (11,3.911)--(10.82,3.911);
\node[gp node right] at (1.5,3.911) {\scriptsize$483$};
\gpcolor{color=gp lt color axes}
\gpsetlinetype{gp lt axes}
\gpsetdashtype{gp dt axes}
\gpsetlinewidth{0.50}
\draw[gp path] (1.504,4.618)--(11,4.618);
\gpcolor{color=gp lt color border}
\gpsetlinetype{gp lt border}
\gpsetdashtype{gp dt solid}
\gpsetlinewidth{1.00}
\draw[gp path] (1.504,4.618)--(1.684,4.618);
\draw[gp path] (11,4.618)--(10.82,4.618);
\node[gp node right] at (1.5,4.618) {\scriptsize$595$};
\gpcolor{color=gp lt color axes}
\gpsetlinetype{gp lt axes}
\gpsetdashtype{gp dt axes}
\gpsetlinewidth{0.50}
\draw[gp path] (1.504,5.718)--(11,5.718);
\gpcolor{color=gp lt color border}
\gpsetlinetype{gp lt border}
\gpsetdashtype{gp dt solid}
\gpsetlinewidth{1.00}
\draw[gp path] (1.504,5.718)--(1.684,5.718);
\draw[gp path] (11,5.718)--(10.82,5.718);
\node[gp node right] at (1.5,5.718) {\scriptsize$769$};
\gpcolor{color=gp lt color axes}
\gpsetlinetype{gp lt axes}
\gpsetdashtype{gp dt axes}
\gpsetlinewidth{0.50}
\draw[gp path] (1.504,6.855)--(11,6.855);
\gpcolor{color=gp lt color border}
\gpsetlinetype{gp lt border}
\gpsetdashtype{gp dt solid}
\gpsetlinewidth{1.00}
\draw[gp path] (1.504,6.855)--(1.684,6.855);
\draw[gp path] (11,6.855)--(10.82,6.855);
\node[gp node right] at (1.5,6.855) {\scriptsize$949$};
\gpcolor{color=gp lt color axes}
\gpsetlinetype{gp lt axes}
\gpsetdashtype{gp dt axes}
\gpsetlinewidth{0.50}
\draw[gp path] (1.504,8.144)--(11,8.144);
\gpcolor{color=gp lt color border}
\gpsetlinetype{gp lt border}
\gpsetdashtype{gp dt solid}
\gpsetlinewidth{1.00}
\draw[gp path] (1.504,8.144)--(1.684,8.144);
\draw[gp path] (11,8.144)--(10.82,8.144);
\node[gp node right] at (1.5,8.144) {\scriptsize$1153$};
\gpcolor{color=gp lt color axes}
\gpsetlinetype{gp lt axes}
\gpsetdashtype{gp dt axes}
\gpsetlinewidth{0.50}
\draw[gp path] (2.200,0.985)--(2.200,8.441);
\gpcolor{color=gp lt color border}
\gpsetlinetype{gp lt border}
\gpsetdashtype{gp dt solid}
\gpsetlinewidth{1.00}
\draw[gp path] (2.200,0.985)--(2.200,1.165);
\draw[gp path] (2.200,8.441)--(2.200,8.261);
\node[gp node center] at (2.200,0.677) {\scriptsize$8$};
\gpcolor{color=gp lt color axes}
\gpsetlinetype{gp lt axes}
\gpsetdashtype{gp dt axes}
\gpsetlinewidth{0.50}
\draw[gp path] (3.128,0.985)--(3.128,8.441);
\gpcolor{color=gp lt color border}
\gpsetlinetype{gp lt border}
\gpsetdashtype{gp dt solid}
\gpsetlinewidth{1.00}
\draw[gp path] (3.128,0.985)--(3.128,1.165);
\draw[gp path] (3.128,8.441)--(3.128,8.261);
\node[gp node center] at (3.128,0.677) {\scriptsize$12$};
\gpcolor{color=gp lt color axes}
\gpsetlinetype{gp lt axes}
\gpsetdashtype{gp dt axes}
\gpsetlinewidth{0.50}
\draw[gp path] (4.057,0.985)--(4.057,8.441);
\gpcolor{color=gp lt color border}
\gpsetlinetype{gp lt border}
\gpsetdashtype{gp dt solid}
\gpsetlinewidth{1.00}
\draw[gp path] (4.057,0.985)--(4.057,1.165);
\draw[gp path] (4.057,8.441)--(4.057,8.261);
\node[gp node center] at (4.057,0.677) {\scriptsize$16$};
\gpcolor{color=gp lt color axes}
\gpsetlinetype{gp lt axes}
\gpsetdashtype{gp dt axes}
\gpsetlinewidth{0.50}
\draw[gp path] (4.985,0.985)--(4.985,8.441);
\gpcolor{color=gp lt color border}
\gpsetlinetype{gp lt border}
\gpsetdashtype{gp dt solid}
\gpsetlinewidth{1.00}
\draw[gp path] (4.985,0.985)--(4.985,1.165);
\draw[gp path] (4.985,8.441)--(4.985,8.261);
\node[gp node center] at (4.985,0.677) {\scriptsize$20$};
\gpcolor{color=gp lt color axes}
\gpsetlinetype{gp lt axes}
\gpsetdashtype{gp dt axes}
\gpsetlinewidth{0.50}
\draw[gp path] (5.913,0.985)--(5.913,8.441);
\gpcolor{color=gp lt color border}
\gpsetlinetype{gp lt border}
\gpsetdashtype{gp dt solid}
\gpsetlinewidth{1.00}
\draw[gp path] (5.913,0.985)--(5.913,1.165);
\draw[gp path] (5.913,8.441)--(5.913,8.261);
\node[gp node center] at (5.913,0.677) {\scriptsize$24$};
\gpcolor{color=gp lt color axes}
\gpsetlinetype{gp lt axes}
\gpsetdashtype{gp dt axes}
\gpsetlinewidth{0.50}
\draw[gp path] (6.842,0.985)--(6.842,8.441);
\gpcolor{color=gp lt color border}
\gpsetlinetype{gp lt border}
\gpsetdashtype{gp dt solid}
\gpsetlinewidth{1.00}
\draw[gp path] (6.842,0.985)--(6.842,1.165);
\draw[gp path] (6.842,8.441)--(6.842,8.261);
\node[gp node center] at (6.842,0.677) {\scriptsize$28$};
\gpcolor{color=gp lt color axes}
\gpsetlinetype{gp lt axes}
\gpsetdashtype{gp dt axes}
\gpsetlinewidth{0.50}
\draw[gp path] (7.770,0.985)--(7.770,8.441);
\gpcolor{color=gp lt color border}
\gpsetlinetype{gp lt border}
\gpsetdashtype{gp dt solid}
\gpsetlinewidth{1.00}
\draw[gp path] (7.770,0.985)--(7.770,1.165);
\draw[gp path] (7.770,8.441)--(7.770,8.261);
\node[gp node center] at (7.770,0.677) {\scriptsize$32$};
\gpcolor{color=gp lt color axes}
\gpsetlinetype{gp lt axes}
\gpsetdashtype{gp dt axes}
\gpsetlinewidth{0.50}
\draw[gp path] (8.698,0.985)--(8.698,8.441);
\gpcolor{color=gp lt color border}
\gpsetlinetype{gp lt border}
\gpsetdashtype{gp dt solid}
\gpsetlinewidth{1.00}
\draw[gp path] (8.698,0.985)--(8.698,1.165);
\draw[gp path] (8.698,8.441)--(8.698,8.261);
\node[gp node center] at (8.698,0.677) {\scriptsize$36$};
\gpcolor{color=gp lt color axes}
\gpsetlinetype{gp lt axes}
\gpsetdashtype{gp dt axes}
\gpsetlinewidth{0.50}
\draw[gp path] (9.626,0.985)--(9.626,8.441);
\gpcolor{color=gp lt color border}
\gpsetlinetype{gp lt border}
\gpsetdashtype{gp dt solid}
\gpsetlinewidth{1.00}
\draw[gp path] (9.626,0.985)--(9.626,1.165);
\draw[gp path] (9.626,8.441)--(9.626,8.261);
\node[gp node center] at (9.626,0.677) {\scriptsize$40$};
\gpcolor{color=gp lt color axes}
\gpsetlinetype{gp lt axes}
\gpsetdashtype{gp dt axes}
\gpsetlinewidth{0.50}
\draw[gp path] (10.555,0.985)--(10.555,8.441);
\gpcolor{color=gp lt color border}
\gpsetlinetype{gp lt border}
\gpsetdashtype{gp dt solid}
\gpsetlinewidth{1.00}
\draw[gp path] (10.555,0.985)--(10.555,1.165);
\draw[gp path] (10.555,8.441)--(10.555,8.261);
\node[gp node center] at (10.555,0.677) {\scriptsize$44$};
\draw[gp path] (1.504,8.441)--(1.504,0.985)--(11,0.985)--(11,8.441)--cycle;
\fill[white, opacity=0.95] (2,6.5) rectangle (4.5,7.5);
\draw[black, very thin] (2,6.5) rectangle (4.5,7.5);
\node[gp node right] at (3,7) {$t[n]$};
\gpcolor{rgb color={0.000,0.000,1.000}}
\gpsetlinewidth{2.00}
\draw[gp path] (3.25,7)--(4.25,7);
\draw[gp path] (2.200,1.162)--(3.128,1.389)--(4.057,1.769)--(4.985,2.344)--(5.913,3.026)%
  --(6.842,3.911)--(7.770,4.618)--(8.698,5.718)--(9.626,6.855)--(10.555,8.144);
\gpsetpointsize{6.00}
\gp3point{gp mark 7}{}{(2.200,1.162)}
\gp3point{gp mark 7}{}{(3.128,1.389)}
\gp3point{gp mark 7}{}{(4.057,1.769)}
\gp3point{gp mark 7}{}{(4.985,2.344)}
\gp3point{gp mark 7}{}{(5.913,3.026)}
\gp3point{gp mark 7}{}{(6.842,3.911)}
\gp3point{gp mark 7}{}{(7.770,4.618)}
\gp3point{gp mark 7}{}{(8.698,5.718)}
\gp3point{gp mark 7}{}{(9.626,6.855)}
\gp3point{gp mark 7}{}{(10.555,8.144)}
\gp3point{gp mark 7}{}{(3.75,7)}
\gpcolor{color=gp lt color border}
\gpsetlinewidth{1.00}
\draw[gp path] (1.504,8.441)--(1.504,0.985)--(11,0.985)--(11,8.441)--cycle;
\node[gp node center,rotate=-270.0] at (0.292,4.713) {$t[n]$};
\node[gp node center] at (6,0.215) {$n$};
%% coordinates of the plot area
\gpdefrectangularnode{gp plot 1}{\pgfpoint{1.504cm}{0.985cm}}{\pgfpoint{11cm}{8.441cm}}
\end{tikzpicture}
%% gnuplot variables

    \vspace{-30pt}
    \caption{Tempo de execução em função do tamanho da entrada}
\end{figure}

Note que o gráfico parece representar uma função quadrática. De fato isso é coerente: fazendo a análise assintótica do algoritmo implementado, ele é composto por duas estruturas de repetição aninhadas que dependem do tamanho da entrada ($N$), ou seja, o algoritmo é $\Theta(N^2)$. A chamada a \cod{SINCOS} dentro desses loops pode causar confusão, pois há estrutura de repetição nessa função, mas note: a quantidade de iterações dessa estrutura é fixa, são sempre vinte e uma iterações.
    
\section{Conclusão}
Nesse laboratório, tivemos intenso contato com o RARS e suas ferramentas, desenvolvemos códigos complexos em Assembly e entendemos melhor a compilação de linguagens de alto nível. Além disso, fizemos análises de desempenho e todas tiveram resultados coerentes.

A experiência obtida neste laboratório, bem como os resultados alcançados e a pesquisa realizada, conduzem à reflexão sobre aspectos cruciais da programação, tais como: a necessidade de compiladores eficientes para que a abstração proporcionada por linguagens de alto nível não sejam demasiado caras; o custo associado ao uso de números de ponto flutuante — que exigem mais instruções para operações aparentemente simples; e a importância de conhecer o \textit{hardware} utilizado para compreender adequadamente o tempo de execução e o desempenho de programas.

Todos os códigos desenvolvidos ou gerados (via compilação cruzada) para a realização desse laboratório foram enviados compactados em um arquivo \cod{.zip}. Para acessar o vídeo solicitado, \href{https://www.youtube.com/watch?v=0igjcjJHzPU}{clique aqui}.
\end{document}